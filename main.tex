\documentclass{article}
\usepackage{mathtools}
\usepackage{graphicx} % Required for inserting images
\usepackage[a4paper, total={6in, 8in}]{geometry}
\usepackage{minted}
\usepackage{amsfonts} % Add this line to include the amsfonts package

\setlength\parindent{0pt}


\title{BIOMATH 205 Answers Almanac}
\author{Simon A. Lee, Zach Schlamowitz}
\date{}

\begin{document}

\maketitle

\section*{Purpose}

A wholistic answer sheet to help us study for the BIOMATH 205 - Computational Algorithms comprehensive exam. Below we will display some rules to ensure fairness and not just people copying from one another.

\begin{itemize}
    \item You must contribute a minimum of one answer per chapter.
    \item tackle different problems from the chapter (no repeated content). 
    \item write the question and answer clearly and explain any steps that are necessary. 
    \item sign off who did the problem so we can contact the person if any logic doesn't make sense.
\end{itemize}

\subsubsection*{I skipped programming questions for the most part since they wont be useful for the qual}

\section*{Chapter 1: Ancient Algorithms}
%% Question 2 %%
\subsubsection*{Q2: Use the significand and exponent functions of Julia and devise a better initial value than $x_0 = 1.0$ for the Babylonian method. Explain your choice, and test it on a few examples.}

We modify the Babylonian method to use an initial guess that is based on the value $c$ whose root we are computing. We begin by claiming that the order of magnitude (in powers of 2) of the square root of $c$ is roughly double that of the root. To see this, consider the following proof:

\textbf{Proof}

Suppose $\sqrt{c} = \alpha \cdot 2^{\beta}$. By definition, $\alpha = \text{significand}(\sqrt{c})$ and $\beta = \text{exponent}(\sqrt{c})$. As such, $\alpha \in [1,2)$. Then:

\begin{align*}
c &= \sqrt{c} \cdot \sqrt{c} \\
&= (\alpha \cdot 2^{\beta}) \cdot (\alpha \cdot 2^{\beta}) \\
&= \alpha^2 \cdot 2^{2\beta}
\end{align*}

If $\alpha^2 \in [1,2)$, then $\text{exponent}(c) = 2\beta = 2 \cdot \text{exponent}(\sqrt{c})$. Otherwise, $\alpha^2 > 2 \implies \text{significand}(c) = \frac{\alpha^2}{2}$ and $\text{exponent}(c) = 2(\text{exponent}(\sqrt{c}) + 1)$. In either case, $\text{exponent}(c) \approx 2 \cdot \text{exponent}(\sqrt{c}$).

We can take advantage of this fact to quicken the algorithm by setting the initial guess at a value that has the same order of magnitude (in powers of 2) as $\sqrt{c}$ so that it starts out (potentially) much closer to $\sqrt{c}$ than $x_0 = 1$ does. We accomplish this by using an initial guess of: $x_0 = \text{significand}(c) \cdot 2^{\frac{\text{exponent}(c)}{2}}$. To make this speed increase even more precise, we could build in if-else logic to determine which subcase of the above proof holds and set the exponent of $x_0$ accordingly, but we did not implement this for clarity's sake. Code for the original Babylonian method and our modified version is shown below, with an example demonstrating the potential efficiency increase by way of the number of required iterations.

\begin{minted}[breaklines]{python}
function babylonian(c::T, tol::T) where T <: Real
    x = one(T)  # start x at 1
    print("Initial guess x_0 is: ")
    println(x)
    print()
    println("Beginning iterations.")
    half = one(T) / 2  # half = 0.5
    for n = 1:100
      print("Iteration (n) = ")
      print(n)
      print(". Current value = ")
      x = half * (x + c / x)
      println(x)
      if abs(x^2 - c) < tol  # convergence test
        println("Tolerance reached. Stopping.")
        return x
      end
    end
  end

babylonian(1736.234^2,1e-10);

>>>Initial guess x_0 is: 1.0
Beginning iterations.
Iteration (n) = 1. Current value = 1.5072547513779998e6
Iteration (n) = 2. Current value = 753628.3756886681
Iteration (n) = 3. Current value = 376816.18784101674
Iteration (n) = 4. Current value = 188412.09389264337
Iteration (n) = 5. Current value = 94214.04672075355
Iteration (n) = 6. Current value = 47123.02155070868
Iteration (n) = 7. Current value = 23593.49629329818
Iteration (n) = 8. Current value = 11860.632457504966
Iteration (n) = 9. Current value = 6057.396656948727
Iteration (n) = 10. Current value = 3277.5270475989328
Iteration (n) = 11. Current value = 2098.638981572482
Iteration (n) = 12. Current value = 1767.5250824162194
Iteration (n) = 13. Current value = 1736.5109782020404
Iteration (n) = 14. Current value = 1736.2340220893864
Iteration (n) = 15. Current value = 1736.234
Tolerance reached. Stopping.
\end{verbatim}

The modified Babylonian method: 
\begin{verbatim}
function babylonian_mod(c::T, tol::T) where T <: Real
    x = significand(c) * 2^(exponent(c)/2)  # start x at 2^exponent(c)
    print("Initial guess x_0 is: ")
    println(x)
    print()
    println("Beginning iterations.")
    half = one(T) / 2  # half = 0.5
    for n = 1:100
      print("Iteration (n) = ")
      print(n)
      print(". Current value = ")
      x = half * (x + c / x)
      println(x)
      if abs(x^2 - c) < tol  # convergence test
        println("Tolerance reached. Stopping.")
        return x
      end
    end
  end

babylonian_mod(1736.234^2,1e-10);

>>> Initial guess x_0 is: 2081.620511956322
Beginning iterations.
Iteration (n) = 1. Current value = 1764.8875999131856
Iteration (n) = 2. Current value = 1736.4666008715867
Iteration (n) = 3. Current value = 1736.2340155785216
Iteration (n) = 4. Current value = 1736.234
Tolerance reached. Stopping.
\end{minted}

For example, using $c = 1736.234^2$, the standard Babylonian algorithm requires 15 iterations to complete, whereas our modified version requires only 4.

\textbf{\textit{- Zach Schlamowitz }} \\\\

\subsubsection*{Q3:  For $c \geq 0$ show that the iteration scheme
\[
x_{n+1} = \frac{c + x_n}{1 + x_n}
\]
converges to $\sqrt{c}$ starting from any $x_0 \geq 0$. Verify either theoretically or empirically that the rate of convergence is much slower than that of the Babylonian method.}

\textbf{Answer:}

To analyze the convergence of the given iteration scheme:
\[
x_{n+1} = \frac{c + x_n}{1 + x_n}
\]
to $\sqrt{c}$ for $c \geq 0$, we can use both theoretical insights and some empirical evidence.

\textbf{Theoretical Analysis}\\
\textbf{1. Fixed Points:} We start by identifying fixed points of the iteration. Setting $x_{n+1} = x_n = x$, we get:
   \[
   x = \frac{c + x}{1 + x} \Rightarrow x + x^2 = c + x \Rightarrow x^2 - x + c = 0.
   \]
   The solutions to this quadratic are:
   \[
   x = \frac{1 \pm \sqrt{1 - 4c}}{2}.
   \]
   However, we need real solutions for $x$, thus we require $1 - 4c \geq 0$, which is not necessarily true for all $c \geq 0$. This calculation does not immediately suggest convergence to $\sqrt{c}$, but we look further.

\textbf{2. Behavior near $\sqrt{c}$:} Instead, consider approximating $x_{n+1} \approx \sqrt{c}$:
   \[
   \sqrt{c} = \frac{c + \sqrt{c}}{1 + \sqrt{c}} \Rightarrow \sqrt{c} + \sqrt{c}^2 = c + \sqrt{c} \Rightarrow \sqrt{c}^2 = c,
   \]
   which is valid, indicating $\sqrt{c}$ could indeed be a fixed point. Let’s analyze local stability by evaluating the derivative of the function at $\sqrt{c}$.

   Define $f(x) = \frac{c + x}{1 + x}$. Then,
   \[
   f'(x) = \frac{(c + x)'(1 + x) - (c + x)(1 + x)'}{(1 + x)^2} = \frac{1 \cdot (1 + x) - (c + x) \cdot 1}{(1 + x)^2} = \frac{1 + x - c - x}{(1 + x)^2} = \frac{1 - c}{(1 + x)^2}.
   \]
   At $x = \sqrt{c}$:
   \[
   f'(\sqrt{c}) = \frac{1 - c}{(1 + \sqrt{c})^2}.
   \]
   If $|f'(\sqrt{c})| < 1$, then $\sqrt{c}$ is an attracting fixed point. Note that for large $c$, the derivative at $\sqrt{c}$ suggests that the convergence could be slow since the denominator grows, making the absolute value of $f'(\sqrt{c})$ smaller.\\

\textbf{Empirical Analysis}
To test empirically, we can compare the convergence rate of this method to the Babylonian method for computing square roots (also known as Newton’s method for $f(x) = x^2 - c$). Newton's iteration scheme is given by:
\[
x_{n+1} = x_n - \frac{x_n^2 - c}{2x_n} = \frac{x_n^2 + c}{2x_n} = \frac{1}{2} \left(x_n + \frac{c}{x_n}\right).
\]
Newton's method typically converges quadratically, meaning the number of correct digits roughly doubles at each step under ideal conditions.

We can compare this by numerically iterating both schemes starting from the same $x_0$ for a specific $c$ and observing the number of iterations required to achieve a certain accuracy. \\\\

\subsubsection*{Q5: Find coefficients $(a, b, c)$ where the standard quadratic formula is grossly inaccurate when implemented in single precision. You will have to look up how to represent single precision numbers in Julia.}

\textbf{Answer:}\\
The standard quadratic formula, $x = \frac{-b \pm \sqrt{b^2 - 4ac}}{2a}$, can be particularly inaccurate in single precision floating point arithmetic under specific conditions. This inaccuracy arises primarily from **catastrophic cancellation**, which occurs when subtracting two close numbers, diminishing the significant digits of the result. A notorious example is when $b^2 \gg 4ac$, making $\sqrt{b^2 - 4ac}$ and $b$ very close in magnitude but opposite in sign.

\textbf{Conditions for Inaccuracy}
To find coefficients $(a, b, c)$ that maximize this inaccuracy:
1. Choose $b$ to be large relative to $a$ and $c$.
2. Ensure $b^2 \approx 4ac$ as closely as possible to maximize cancellation.

\textbf{Julia Representation}
In Julia, single precision is represented using `Float32`. A standard way to define these numbers is by appending `f0` to the number, e.g., `1.0f0` for the float version of 1.0.

\textbf{Example Coefficients}
Let's choose:
- \( a = 1.0f0 \)
- \( b = 10^6f0 \) (a large number to enhance the effect of $b^2$)
- \( c = 2.5 \times 10^{11}f0 \) (designed so $4ac \approx b^2$)

Here, \( b^2 = (10^6)^2 = 10^{12} \) and \( 4ac = 4 \times 1.0 \times 2.5 \times 10^{11} = 10^{12} \), fulfilling the condition \( b^2 \approx 4ac \).

\textbf{Calculating the Roots}
Plugging these into the quadratic formula:
- \( \sqrt{b^2 - 4ac} = \sqrt{10^{12} - 10^{12}} = 0 \) (approximation due to limited precision)
- \( x = \frac{-10^6 \pm 0}{2} = -500000 \) (only one root due to cancellation)

\textbf{Derivation}
This calculation shows how a single root appears due to catastrophic cancellation, a gross inaccuracy since theoretically, there should be two distinct roots for different signs in $\pm$. Hence, $(a, b, c) = (1.0f0, 10^6f0, 2.5 \times 10^{11}f0)$ provides a specific case of gross inaccuracy in the quadratic formula when implemented in single precision in Julia. \\



%% QUESTION 6 %%
 \subsubsection*{Q6: Why does the product of the two roots of a quadratic equal $\frac{c}{a}$}

\textbf{Answer:}

Consider the quadratic equation in standard form:
\[ ax^2 + bx + c = 0 \]

\textbf{Step 1: Factorize the quadratic equation }(assuming \(a \neq 0\)):
\[ x^2 + \frac{b}{a}x + \frac{c}{a} = 0 \]

\textbf{Step 2: Use Vieta's Formulas}, which relate the coefficients of a polynomial to sums and products of its roots. For a quadratic equation \(x^2 + px + q = 0\), the roots \(\alpha\) and \(\beta\) satisfy:
\[ \alpha + \beta = -p \quad \text{and} \quad \alpha\beta = q \]

\textbf{Step 3: Apply Vieta's to our equation:}
In the case of the quadratic \(x^2 + \frac{b}{a}x + \frac{c}{a} = 0\):
\[ p = \frac{b}{a} \quad \text{and} \quad q = \frac{c}{a} \]

Therefore, by Vieta's formulas:
\[ \alpha\beta = \frac{c}{a} \]

Thus, the product of the roots \(\alpha\) and \(\beta\) of the quadratic equation \(ax^2 + bx + c = 0\) is \(\frac{c}{a}\), proving the statement.

% Q7
\subsubsection*{Q7: Solving a cubic equation $ax^3 + bx^2 + cx + d = 0$ is much more complicated than solving a quadratic. Demonstrate that (a) the substitution $x = y - \frac{b}{3a}$ reduces the cubic to $y^3 + ey + f = 0$ for certain coefficients $e$ and $f$, (b) the further substitution $y = z - \frac{e}{3z}$ reduces this equation to $z^6 + fz^3 - \frac{e^3}{27} = 0$, and (c) the final substitution $w = z^3$ reduces the equation in $z$ to a quadratic in $w$, which can be explicitly solved. One can now reverse these substitutions and capture six roots, which collapse in pairs to at most three unique roots. Program your algorithm in Julia, and make sure that it captures complex as well as real roots.}

To solve the cubic equation \(ax^3 + bx^2 + cx + d = 0\) using the substitutions described in the problem, let's perform the transformations step by step.

\textbf{Step (a): Substitution \(x = y - \frac{b}{3a}\)}

Substitute \(x = y - \frac{b}{3a}\) into the cubic equation. We aim to eliminate the \(y^2\) term:

\[
a\left(y - \frac{b}{3a}\right)^3 + b\left(y - \frac{b}{3a}\right)^2 + c\left(y - \frac{b}{3a}\right) + d = 0
\]

Expanding and simplifying while collecting terms free of \(y^2\) results in a new equation of the form:

\[
y^3 + Py + Q = 0
\]

Here, \(P\) and \(Q\) are new coefficients derived from \(a, b, c, d\), specifically:
\[
P = c - \frac{b^2}{3a} \quad \text{and} \quad Q = \frac{2b^3}{27a^2} - \frac{bc}{3a} + d
\]

\textbf{Step (b): Substitution \(y = z - \frac{P}{3z}\)}

Next, substitute \(y = z - \frac{P}{3z}\) into the transformed cubic equation. This substitution aims to simplify \(y^3 + Py + Q = 0\) further into a form involving \(z\):

\[
\left(z - \frac{P}{3z}\right)^3 + P\left(z - \frac{P}{3z}\right) + Q = 0
\]

Multiplying through and simplifying, this leads to an equation of the form:

\[
z^6 + Qz^3 - \frac{P^3}{27} = 0
\]

\textbf{Step (c): Substitution \(w = z^3\)}

Finally, let \(w = z^3\). This turns the equation into a quadratic in terms of \(w\):

\[
w^2 + Qw - \frac{P^3}{27} = 0
\]

This quadratic equation can be solved using the standard formula:

\[
w = \frac{-Q \pm \sqrt{Q^2 + 4 \cdot \frac{P^3}{27}}}{2}
\]

\textbf{Finding Roots}

1. Solve for \(w\) using the quadratic formula.
2. Back-substitute to find \(z\) (and hence \(y\), then \(x\)) using \(z = \sqrt[3]{w}\).
3. This results in up to six values for \(z\), but pairs will collapse to three unique roots for \(y\), and hence three unique roots for \(x\).

\textbf{Programming in Julia}

Here’s a simple outline of how you might code this in Julia, capturing complex and real roots:

\begin{minted}[breaklines]{julia}
function solve_cubic(a, b, c, d)
    P = c - b^2 / (3a)
    Q = 2b^3 / (27a^2) - bc / (3a) + d
    discriminant = Q^2 + 4 * P^3 / 27

    w1 = (-Q + sqrt(discriminant)) / 2
    w2 = (-Q - sqrt(discriminant)) / 2

    z_roots = [w1^(1/3), -w1^(1/3), w2^(1/3), -w2^(1/3)]
    y_roots = [z - P / (3z) for z in z_roots]
    x_roots = y_roots .- b / (3a)

    return x_roots
end
\end{minted}

\subsubsection*{Q9: The prime number theorem says that the number of primes $\pi(n)$ between $1$ and $n$ is asymptotic to $\frac{n}{\ln n}$. Use the Sieve of Eratosthenes to check how quickly the ratio $\frac{\pi(n) \ln(n)}{n}$ tends to $1$.}

\begin{minted}[breaklines]{julia}
function sieve(n)
    is_prime = trues(n)
    is_prime[1] = false  # 1 is not a prime number
    for i in 2:sqrt(n)
        if is_prime[i]
            for multiple in i*i:i:n
                is_prime[multiple] = false
            end
        end
    end
    return count(is_prime)  # Returns the number of primes up to n
end
\end{minted}

\subsubsection*{Q11: Show that the perimeter lengths $a_n$ and $b_n$ in Archimedes’ algorithm satisfy
\[ a_n = m \tan \frac{\pi}{m} \quad\] and\[\quad b_n = m \sin \frac{\pi}{m}, \]
where \( m = 2 \cdot 2^n \) is the number of sides of the two regular polygons. Use this representation and appropriate trigonometric identities to prove the recurrence relations (1.3) and (1.4).}

To prove the recurrence relations given the perimeter lengths, we begin with:

- \( a_n = m \tan \left( \frac{\pi}{m} \right) \)
- \( b_n = m \sin \left( \frac{\pi}{m} \right) \)

Where \( m = 2 \cdot 2^n \).

\textbf{Recurrence Relation 1.3:}

\[ a_{n+1} = \frac{2a_nb_n}{a_n + b_n} \]

Given the definitions:
- \( a_n = m \tan \left( \frac{\pi}{m} \right) \)
- \( b_n = m \sin \left( \frac{\pi}{m} \right) \)

Substitute into the recurrence relation:
\[ a_{n+1} = \frac{2 \cdot m \tan \left( \frac{\pi}{m} \right) \cdot m \sin \left( \frac{\pi}{m} \right)}{m \tan \left( \frac{\pi}{m} \right) + m \sin \left( \frac{\pi}{m} \right)} \]
\[ a_{n+1} = \frac{2m^2 \tan \left( \frac{\pi}{m} \right) \sin \left( \frac{\pi}{m} \right)}{m \left(\tan \left( \frac{\pi}{m} \right) + \sin \left( \frac{\pi}{m} \right)\right)} \]
\[ a_{n+1} = \frac{2m \tan \left( \frac{\pi}{m} \right) \sin \left( \frac{\pi}{m} \right)}{\tan \left( \frac{\pi}{m} \right) + \sin \left( \frac{\pi}{m} \right)} \]

This form illustrates that \( a_{n+1} \) is dependent on \( \tan \) and \( \sin \) values, scaled by \( m \), but needs further simplification or numerical computation to directly connect it to the tangent and sine of angles for \( m \) in the next iteration.

\textbf{Recurrence Relation 1.4:}

\[ b_{n+1} = \sqrt{a_{n+1}b_n} \]

Using our previous result:
\[ b_{n+1} = \sqrt{\left(\frac{2m \tan \left( \frac{\pi}{m} \right) \sin \left( \frac{\pi}{m} \right)}{\tan \left( \frac{\pi}{m} \right) + \sin \left( \frac{\pi}{m} \right)}\right) \cdot m \sin \left( \frac{\pi}{m} \right)} \]
\[ b_{n+1} = m \sqrt{\frac{2 \tan \left( \frac{\pi}{m} \right) \sin^2 \left( \frac{\pi}{m} \right)}{\tan \left( \frac{\pi}{m} \right) + \sin \left( \frac{\pi}{m} \right)}} \]

Here, \( b_{n+1} \) is derived from the geometric mean of \( a_{n+1} \) and \( b_n \), also indicating dependence on \( \tan \) and \( \sin \) values of \( \frac{\pi}{m} \). 

These formulas confirm the relationship between the perimeter lengths and trigonometric functions, highlighting the complexity of Archimedes' algorithm when applied recursively through these geometric transformations.

%q12
\subsubsection*{Q12: Based on the trigonometric representations of the previous problem, show that $\frac{1}{3}a_n + \frac{2}{3}b_n$ is a much better approximation to $\pi$ than either $a_n$ or $b_n$ [142]. Check your theoretical conclusions by writing a Julia program that tracks all three approximations to $\pi$.}

To demonstrate why \( \frac{1}{3}a_n + \frac{2}{3}b_n \) is a much better approximation to \( \pi \) than either \( a_n \) or \( b_n \) alone, consider the following:

- \( a_n = m \tan \left(\frac{\pi}{m}\right) \)
- \( b_n = m \sin \left(\frac{\pi}{m}\right) \)

Given that the tangent function overestimates the angle for large angles while the sine function underestimates it, the weighted average of these using \( \frac{1}{3} \) and \( \frac{2}{3} \) potentially balances the overestimation and underestimation.

\textbf{Theoretical Basis}

Both \( \tan(\theta) \) and \( \sin(\theta) \) are approximations of \( \theta \) for small values of \( \theta \), but:
- \( \tan(\theta) \) grows faster than \( \theta \) as \( \theta \) increases.
- \( \sin(\theta) \) grows slower than \( \theta \) as \( \theta \) approaches \( \frac{\pi}{2} \).

Thus, combining them in a weighted manner can potentially yield a more accurate estimate of \( \pi \) compared to using either alone.

\textbf{Julia Program to Check the Approximations}

Below is a simple Julia program to compute \( a_n \), \( b_n \), and \( \frac{1}{3}a_n + \frac{2}{3}b_n \) and compare these to \( \pi \):

\begin{minted}[breaklines]{julia}
using Plots

function pi_approximations(n)
    m = 2 * 2^n
    a_n = m * tan(π / m)
    b_n = m * sin(π / m)
    mixed = (1/3) * a_n + (2/3) * b_n
    
    return a_n, b_n, mixed
end

function track_approximations(n)
    approximations = [pi_approximations(i) for i in 1:n]
    a_ns = [x[1] for x in approximations]
    b_ns = [x[2] for x in approximations]
    mixed = [x[3] for x in approximations]

    plot(1:n, [a_ns, b_ns, mixed], label=["a_n" "b_n" "Mixed"], xlabel="n", ylabel="Value", title="Approximations to Pi")
end

# Track and plot approximations for n = 1 to 10
track_approximations(10)
\end{minted}

\textbf{Expected Outcome}

Upon running this Julia program, the plot will show the convergence of \( a_n \), \( b_n \), and \( \frac{1}{3}a_n + \frac{2}{3}b_n \) to \( \pi \). The mixed approximation should converge closer to \( \pi \) than either \( a_n \) or \( b_n \) alone, especially for larger \( n \), confirming the theoretical insights with empirical evidence.

\subsubsection*{Q13: Consider evaluation of the polynomial
\[ p(x) = a_0x^n + a_1x^{n-1} + \cdots + a_{n-1}x + a_n \]
for a given value of \( x \). If one proceeds naively, then it takes \( n-1 \) multiplications to form the powers \( x^k = x \cdot x^{k-1} \) for \( 2 \leq k \leq n \), \( n \) multiplications to multiply each power \( x^k \) by its coefficient \( a_{n-k} \), and \( n \) additions to sum the resulting terms. This amounts to \( 3n - 1 \) operations in all. A more efficient method exploits the fact that \( p(x) \) can be expressed as 
\[ p(x) = x(a_0x^{n-1} + a_1x^{n-2} + \cdots + a_{n-1}) + a_n \]
\[ = x b_{n-1}(x) + a_n. \]
Since the polynomial \( b_{n-1}(x) \) of degree \( n - 1 \) can be similarly reduced, a complete recursive scheme for evaluating \( p(x) \) is given by 
\[ b_0(x) = a_0, \quad b_k(x) = x b_{k-1}(x) + a_k, \quad k = 1, \ldots, n. \]
This scheme requires only \( n \) multiplications and \( n \) additions in order to compute \( p(x) = b_n(x). \) Program the scheme and extend it to the simultaneous evaluation of the derivative \( p'(x) \) of \( p(x). \)}\\

The efficient method of evaluating the polynomial \(p(x)\) and its derivative \(p'(x)\) can be implemented using Horner's method. This method requires only \(n\) multiplications and \(n\) additions, significantly reducing computational overhead compared to the naive method.

\textbf{Implementing Horner's Method in Julia:}

Below is a simple Julia program that uses Horner's method to evaluate both \(p(x)\) and \(p'(x)\):

\begin{minted}[breaklines]{julia}
function evaluate_polynomial_and_derivative(coeffs, x)
    p = coeffs[end]
    dp = 0  # Derivative initialization
    
    # Horner's method loop for both p(x) and p'(x)
    for i in length(coeffs)-1:-1:1
        dp = dp * x + p
        p = p * x + coeffs[i]
    end
    
    return p, dp
end

# Example coefficients of the polynomial p(x) = 2x^3 - 6x^2 + 2x - 1
coeffs = [2, -6, 2, -1]
x = 3  # Value at which to evaluate the polynomial

# Evaluating the polynomial and its derivative
p, dp = evaluate_polynomial_and_derivative(coeffs, x)

println("The polynomial evaluated at x = $x is $p")
println("The derivative of the polynomial at x = $x is $dp")
\end{minted}


\begin{itemize}
    \item \textbf{1. Horner's Scheme:} The function `evaluate\_polynomial\_and\_derivative` calculates the value of the polynomial and its derivative at a specific point \(x\) using Horner's method. This method evaluates the polynomial efficiently by nesting each term inside the next, reducing the number of necessary operations.
    \item \textbf{2. Recursive Calculation:} The loop computes the polynomial value \(p\) and its derivative \(dp\) simultaneously. Each iteration incorporates the next coefficient from the polynomial into the running total for \(p\) and \(dp\), adjusting for the current power of \(x\).
    \item \textbf{3. Efficiency:} This approach is significantly more efficient than calculating each term separately and then summing them up, as it consolidates the operations into a single traversal of the coefficients.
\end{itemize}

The output of the program provides the evaluated polynomial and its derivative at the chosen point \(x\), demonstrating the practical application of Horner's method.

\subsubsection*{Q14: Consider a sequence \(x_1, \ldots, x_n\) of \(n\) real numbers. After you have computed the sample mean and variance
\[ \mu_n = \frac{1}{n} \sum_{i=1}^n x_i \quad \text{and} \quad \sigma_n^2 = \frac{1}{n} \sum_{i=1}^n (x_i - \mu_n)^2, \]
\text{suppose you are presented with a new observation \(x_{n+1}\). It is possible to adjust the sample mean and variance without revisiting all of the previous observations. Verify theoretically and then code the updates}
\[ \mu_{n+1} = \frac{1}{n+1} (n\mu_n + x_{n+1}) \]
\[ \sigma_{n+1}^2 = \frac{n}{n+1} \sigma_n^2 + \frac{1}{n} (x_{n+1} - \mu_n)^2. \]}

\textbf{Theoretical Verification}

Let's start by verifying the formulas for updating the mean and variance when a new data point is added.

\textbf{Updating the Mean:}

Given the current mean \(\mu_n\):
\[ \mu_n = \frac{1}{n} \sum_{i=1}^n x_i \]

When a new observation \(x_{n+1}\) is added:
\[ \mu_{n+1} = \frac{1}{n+1} (x_1 + x_2 + \ldots + x_n + x_{n+1}) \]
\[ = \frac{1}{n+1} \left( n\mu_n + x_{n+1} \right) \]
\[ = \frac{n}{n+1} \mu_n + \frac{1}{n+1} x_{n+1} \]

This matches the given update formula:
\[ \mu_{n+1} = \frac{1}{n+1} (n\mu_n + x_{n+1}) \]

\textbf{Updating the Variance:}

Given the current variance \(\sigma_n^2\):
\[ \sigma_n^2 = \frac{1}{n} \sum_{i=1}^n (x_i - \mu_n)^2 \]

When a new observation \(x_{n+1}\) is added, the updated variance \(\sigma_{n+1}^2\) is calculated using the new mean \(\mu_{n+1}\) and includes the new observation \(x_{n+1}\):
\[ \sigma_{n+1}^2 = \frac{1}{n+1} \sum_{i=1}^{n+1} (x_i - \mu_{n+1})^2 \]

This can be rearranged and simplified (by expanding the square and using linearity of sums) to:
\[ \sigma_{n+1}^2 = \frac{n}{n+1} \sigma_n^2 + \frac{1}{n+1} (x_{n+1} - \mu_n)^2 \]

This formula reflects how the variance needs to account for the distance of the new observation from the previous mean, not the new mean, to avoid recalculating terms involving the old observations. The term \(\frac{1}{n}\) adjusts for the change in the number of observations in the variance calculation.

\textbf{Julia Code Implementation}

Here's a simple Julia function to update the mean and variance based on these formulas:

\begin{minted}[breaklines]{julia}
function update_stats(current_mean, current_variance, n, new_value)
    new_mean = (n * current_mean + new_value) / (n + 1)
    new_variance = (n * current_variance / (n + 1)) + ((new_value - current_mean)^2 / (n + 1))
    return new_mean, new_variance
end

# Example usage
n = 100  # Number of existing observations
current_mean = 50.0  # Current mean of observations
current_variance = 25.0  # Current variance of observations
new_value = 48.0  # New observation

new_mean, new_variance = update_stats(current_mean, current_variance, n, new_value)
println("Updated Mean: $new_mean")
println("Updated Variance: $new_variance")
\end{minted}

\newpage
\section*{Chapter 2: Sorting}

\textbf{Q2}: Show the worst case of quicksort takes on the order of \(n^2\) operations.

\textbf{Answer:}

To demonstrate that the worst-case scenario for the QuickSort algorithm takes on the order of \(n^2\) operations, we can construct an example and analyze its behavior. The worst-case scenario occurs when the pivot chosen at each step consistently divides the array into two unbalanced subarrays, one of size 
\(n-1\) and the other of size 0.\\

Here is an example that showcases the behavior. \\ 

Consider an array of $n$ elements sorted in ascending order: \\ 

\(1,2,3,4,5,...,n-1,n\) \\

In this case, if we chose the lst element as the pivot, the partitioning step will result in one subarray of size \(n-1\), and one subarray of size 0. \\

Now we have reduced the array with one less element. If we continue to always choose the last element as the pivot this behavior will repeat reducing our array size by a single entry. \\

At each step, we have $n$ elements, and we perform $n$ comparisons to partition them. Since we repeat this process $n$ times (each time with \(n-1\) elements, then \(n-2\) and so on), the total number of comparisons become: \\ 

\(n + (n-1) + (n-2) + ... + 1 \) which is equal to the arithmetic sequence \(\frac{n(n+1)}{2}\) \\

Therefore, the number of comparisons in the worst case scenario is on the order of \(\frac{n(n+1)}{2}\) which is proportional to $n^2$ \\

From this example we see why worst case scenario would result in $n^2$ operations. \\

\textbf{\textit{- Simon Lee }} \\\\

\textbf{Q4} Given a sorted array of numbers of lengthnand a number $c$, write a Julia program to find the pair of numbers in the array whose sum is closest to $c$. An efficient solution can find the pair in $O(n)$ time?\\

\textbf{Answer}

\begin{verbatim}
    function find_closest_pair(arr::Vector{Int}, c::Int)
        if length(arr) < 2
            return (0, 0)  # Not enough elements for a pair
        end
        
        left, right = 1, length(arr)
        closest_diff = abs(arr[left] + arr[right] - c)
        closest_pair = (arr[left], arr[right])
        
        while left < right
            current_diff = abs(arr[left] + arr[right] - c)
            if current_diff < closest_diff
                closest_diff = current_diff
                closest_pair = (arr[left], arr[right])
            end
            
            if arr[left] + arr[right] < c
                left += 1
            else
                right -= 1
            end
        end
        
        return closest_pair
    end
    
    # Example of the question
    arr = [1, 2, 4, 7, 10, 14]
    c = 8
    result = find_closest_pair(arr, c)
    println("Closest pair to $c is $result")

    >>> Closest pair to 8 is (1, 7)
\end{verbatim}

\textbf{\textit{- Simon Lee }} \\\\

\textbf{Q6:} Suppose you are given two ordered arrays $x$ and $y$ of integers. If these represent integer sets, write Julia functions to find their union, intersection, and set difference. Do not use existing Julia functions for set operations 

\textbf{Answer} 
\begin{verbatim}
function set_union(x::Array, y::Array)
    """
    Compute the set union of two sets that are represented as ordered (1xN) integer arrays.
    Takes advantage of the ordered property of the arrays by identifying elements in common
    by moving along the elements of each input array and assembling the union in order. 
    RETURNS the set union as an Array matching the input types.
    """
    z = Array{Int64}(undef, 1, size(x, 2) + size(y, 2))
    # Initialize indices
    i = 1 # index for set x
    j = 1 # index for set y
    k = 1 # index for set z (union set)

    # Iterate down the input arrays simultaneously keeping track of union's order
    while i <= length(x) && j <= length(y)
        # Identify next element in union's order and insert into union
        if x[i] < y[j]
            z[k] = x[i]
            i = i + 1
        elseif x[i] > y[j]
            z[k] = y[j]
            j = j + 1
        elseif x[i] == y[j]
            z[k] = x[i]
            i = i + 1
            j = j + 1
        end
        k = k + 1
    end
    
    # Put any remaining entries into union
    if i > length(x)
        z[k:k+(length(y)-j)] = y[j:end] # put rest of y into z
        k = k + (length(y)-j)
    elseif j > length(y)
        z[k:k+(length(x)-i)] = x[i:end] # put rest of x into z
        k = k+(length(x)-i)
    end

    # If there were duplicates, pre-allocated size of z will have been 
    # too long, so we now put the actual values into an output array
    union_array = z[1:k]

    return Array(union_array')
end

function set_intersection(x::Array, y::Array)
    """
    Compute the set intersection of two sets that are represented as ordered (1xN) integer arrays.
    Takes advantage of the ordered property of the arrays by identifying elements in common
    by moving along the elements of each input array and assembling the intersection in order. 
    RETURNS the set intersection as an Array to match the input types.
    """
    z = Int64[] # Initialize intersection as a vector for ease of internal use

    # Initialize indices
    i = 1 # index for set x
    j = 1 # index for set y

    # Iterate down the input arrays simultaneously keeping track of intersection's order
    while i <= length(x) && j <= length(y)
        # Identify next element in union's order and insert into intersection
        if x[i] < y[j]
            i = i + 1
        elseif x[i] > y[j]
            j = j + 1
        elseif x[i] == y[j]
            push!(z, x[i])
            i = i + 1
            j = j + 1
        end
    end

    return Array(z')
end

function set_difference(x::Array, y::Array)
    """
    Compute the set difference (x - y) of two sets (x,y) that are represented as ordered (1xN) 
    integer arrays. Takes advantage of the ordered property through use of the above-defined
    intersection function. 
    RETURNS the set difference as an Array to match the input types.
    """
    # Initialize set difference as a vector for ease of internal use
    z = Int64[]

    # Obtain set intersection to identify shared elements
    inter = set_intersection(x,y)

    # Iterate along array x, putting elements into the output array z which are not in y
    i = 1 # Index for set x
    j = 1 # Index for intersection array
    while j <= length(inter)
        if x[i] < inter[j]
            push!(z, x[i])
            i+=1
        else # shared element! (Because inter[j]>x[i] can't happen, since inter is a subset of x)
            i+=1
            j+=1
        end
    end

    # Add any remaining elements of x to z
    append!(z,x[i:end])

    return Array(z')

end
    
\end{verbatim}

And now some examples of their use:

Example 1:
\begin{verbatim}
a = [-1 0 13 22 50 64 78 88 89 90 91 92] # notice presence of duplicates
b = [20 22 64 75 114 116]
c = set_union(a,b)
print("Union = ")
println(c)
print()

d = set_intersection(a,b)
print("Intersection = ")
println(d)
print()

e = set_difference(a,b)
print("Set Difference = ")
println(e)

>>> Union = [-1 0 13 20 22 50 64 75 78 88 89 90 91 92 114 116]
Intersection = [22 64]
Set Difference = [-1 0 13 50 78 88 89 90 91 92]
\end{verbatim}

Example 2:
\begin{verbatim}
A = [-31 -13 0 11 13 14]
B = [-30 -24 -13 11 13 14]
print("Union = ")
println(set_union(A,B))
print()

print("Intersection = ")
println(set_intersection(A,B))
print()

print("Set Difference A - B = ")
println(set_difference(A,B))
print("Set Difference B - A = ")
println(set_difference(B,A))
print()

>>> Union = [-31 -30 -24 -13 0 11 13 14]
Intersection = [-13 11 13 14]
Set Difference A - B = [-31 0]
Set Difference B - A = [-30 -24]
\end{verbatim}

\textbf{\textit{- Zach Schlamowitz }} \\\\

\newpage

\section*{Chapter 5: Solutions of Linear Equations}

\subsubsection*{Q2: Verify our contention that it takes about \(\frac{2}{3} n^3\) arithmetic operations to form the LU decomposition of an \(n \times n\) matrix.}

To verify the \(\frac{2}{3} n^3\) arithmetic operations claim for the LU decomposition of an \(n \times n\) matrix, let's perform the actual calculation by analyzing the operation count required in each step of the decomposition.

\textbf{Operation Count Analysis for LU Decomposition}

\textbf{1. Gaussian Elimination Process:}
   In LU decomposition, Gaussian elimination is used to form the lower triangular (L) and upper triangular (U) matrices. Here's how the operations break down for each step:

\textbf{2. Eliminating Elements:}
   For each pivot (diagonal element) from the first row/column to the \((n-1)\)-th row/column:
   - To eliminate entries below each pivot, you need to perform subtraction operations on rows below the pivot row.
   - For each pivot in the \(i\)-th column:
     - There are \(n-i\) elements to eliminate (below the pivot).
     - For each element, you update \(n-i\) elements in the row.
     - Each update requires a multiplication and a subtraction.

\textbf{3. Calculating Operations:}
   The total number of arithmetic operations for eliminating elements below each pivot is calculated as:
   \[
   \text{Operations for column } i = (n-i) \times (n-i) \times 2
   \]
   where \( (n-i) \) is the number of elements to eliminate, and each element requires \( (n-i) \) multiplications and \( (n-i) \) subtractions.

\textbf{4. Summing Up Operations:}
   Summing the operations for each column from 1 to \(n-1\) gives:
   \[
   \text{Total operations} = \sum_{i=1}^{n-1} 2(n-i)^2 = 2\sum_{k=1}^{n-1} k^2
   \]
   The sum of squares of the first \(n-1\) integers is:
   \[
   \sum_{k=1}^{n-1} k^2 = \frac{(n-1)n(2n-1)}{6}
   \]
   Substituting into the total operations formula:
   \[
   \text{Total operations} = 2 \times \frac{(n-1)n(2n-1)}{6}
   \]

\textbf{5. Simplifying to Show \(\frac{2}{3} n^3\):}
   For large \(n\), the formula simplifies (by approximating \(n-1 \approx n\)):
   \[
   \text{Total operations} \approx 2 \times \frac{n \times n \times (2n)}{6} = \frac{2n^3}{3}
   \]

This theoretical analysis shows that the number of operations required for LU decomposition indeed approaches \(\frac{2}{3} n^3\) for a large \(n\), which aligns with the contention. This count primarily accounts for the multiplications and subtractions required to zero out the matrix elements below each pivot, considering each pivot's influence on the subsequent rows and columns during the elimination process.

\subsubsection*{Q3: Prove that (a) the product of two upper-triangular matrices is upper triangular, (b) the inverse of an upper-triangular matrix is upper triangular, (c) if the diagonal entries of an upper-triangular matrix are positive, then the diagonal entries of its inverse are positive, and (d) if the diagonal entries of an upper-triangular matrix are unity, then the diagonal entries of its inverse are unity. Similar statements apply to lower-triangular matrices.}

Here we'll prove the properties of upper-triangular matrices step-by-step, considering each sub-question.

\textbf{(a)} The product of two upper-triangular matrices is upper triangular

\textbf{Proof:}
Let \( A \) and \( B \) be two upper-triangular \( n \times n \) matrices, meaning all entries below the diagonal in both matrices are zero, i.e., \( A_{ij} = 0 \) and \( B_{ij} = 0 \) for \( i > j \). The product \( C = AB \) has entries \( C_{ij} \) defined as:
\[ C_{ij} = \sum_{k=1}^n A_{ik}B_{kj} \]
For \( i > j \), \( A_{ik} = 0 \) for \( k < i \) (since \( A \) is upper triangular) and \( B_{kj} = 0 \) for \( k > j \) (since \( B \) is upper triangular). Since \( i > j \), all terms in the sum involve a zero from either \( A_{ik} \) or \( B_{kj} \), making \( C_{ij} = 0 \) for \( i > j \). Thus, \( C \) is upper triangular.

\textbf{(b)} The inverse of an upper-triangular matrix is upper triangular

\textbf{Proof:}
Consider an upper-triangular matrix \( A \) with non-zero diagonal entries (ensuring it's invertible). The inverse \( A^{-1} \) can be found such that \( AA^{-1} = I \). Since the diagonal entries are non-zero, \( A^{-1} \) must "undo" the multiplicative effect of \( A \), which can be achieved only if \( A^{-1} \) is also upper-triangular. This is because any lower components would result in non-zero entries below the diagonal in the product \( I \).

\textbf{(c)} If the diagonal entries of an upper-triangular matrix are positive, then the diagonal entries of its inverse are positive

\textbf{Proof:}
For an upper-triangular matrix \( A \) with positive diagonal entries, each diagonal element of \( A^{-1} \), say \( (A^{-1})_{ii} \), is computed as the reciprocal of the corresponding diagonal element of \( A \) (in the simplest case, without off-diagonal elements). Since the reciprocal of a positive number is positive, \( (A^{-1})_{ii} > 0 \).

\textbf{(d)} If the diagonal entries of an upper-triangular matrix are unity, then the diagonal entries of its inverse are unity

\textbf{Proof:}
For an upper-triangular matrix \( A \) where all diagonal entries are 1 (\( A_{ii} = 1 \)), each corresponding diagonal entry of \( A^{-1} \) must also be 1 to satisfy \( AA^{-1} = I \). Specifically, \( (A^{-1})_{ii} \cdot A_{ii} = 1 \cdot 1 = 1 \), ensuring that \( (A^{-1})_{ii} = 1 \).

\subsubsection*{Q4: Demonstrate that an orthogonal upper-triangular matrix is diagonal.}

To demonstrate that an orthogonal upper-triangular matrix is diagonal, consider an orthogonal matrix \( Q \) that is also upper-triangular. Recall that an orthogonal matrix satisfies \( Q^TQ = I \), where \( I \) is the identity matrix.

\textbf{Step-by-Step Proof}

\textbf{1. Orthogonality and Upper-Triangular Definition:}
   Let \( Q \) be an upper-triangular matrix. This means all entries below the diagonal are zero:
   \[
   Q = \begin{bmatrix}
   q_{11} & q_{12} & \cdots & q_{1n} \\
   0 & q_{22} & \cdots & q_{2n} \\
   \vdots & \vdots & \ddots & \vdots \\
   0 & 0 & \cdots & q_{nn}
   \end{bmatrix}
   \]

\textbf{2. Orthogonality Condition:}
   The matrix \( Q \) being orthogonal implies \( Q^TQ = I \). This leads to specific conditions on the elements of \( Q \):
   \[
   (Q^TQ)_{ij} = \sum_{k=1}^n Q^T_{ik}Q_{kj} = \delta_{ij}
   \]
   where \( \delta_{ij} \) is the Kronecker delta, which is 1 if \( i=j \) and 0 otherwise.

\textbf{3. Evaluating \( Q^TQ \):}
   The product \( Q^TQ \) for an upper-triangular \( Q \) can be written explicitly. Note that \( Q^T \) is lower-triangular. The \( (i, j) \) entry of \( Q^TQ \) involves summing the products of elements in the \( i \)-th row of \( Q^T \) and the \( j \)-th column of \( Q \):
   \[
   (Q^TQ)_{ij} = \sum_{k=\max(i,j)}^n Q^T_{ik}Q_{kj}
   \]
   This sum is non-zero only when \( i = j \) due to the zero entries below the diagonal in \( Q^T \) and above the diagonal in \( Q \).

\textbf{4. Zero Non-Diagonal Elements:}
   For \( i \neq j \), \( Q^T_{ik}Q_{kj} \) involves products of elements where at least one of the terms is zero due to the triangular structure. Thus, \( (Q^TQ)_{ij} = 0 \) for all \( i \neq j \).

\textbf{5. Diagonal Elements:}
   The diagonal entries of \( Q^TQ \) (i.e., \( (Q^TQ)_{ii} \)) equate to the sum of squares of the elements in the \( i \)-th row/column of \( Q \) due to orthogonality:
   \[
   (Q^TQ)_{ii} = \sum_{k=i}^n Q_{ki}^2 = 1
   \]
   Since \( Q \) is upper-triangular, this simplifies to \( Q_{ii}^2 = 1 \), implying \( Q_{ii} = \pm 1 \) and all other off-diagonal entries in the same row and column must be zero to meet the orthogonality condition.

\text{Thus, an orthogonal upper-triangular matrix must have zeros in all off-diagonal entries, proving it is diagonal.}

\subsubsection*{Q5: Prove that the set of permutation matrices \(P\) forms a finite group closed under the formation of products and inverses. How many \(n \times n\) permutation matrices exist? Recall that a permutation \(\sigma\) is a one-to-one map of the set \(\{1, 2, \ldots, n\}\) onto itself.}

\textbf{Proof and Count of Permutation Matrices}

\textbf{Proof that Permutation Matrices Form a Group}

\textbf{1. Closure}: The product of two permutation matrices is another permutation matrix. This is because the multiplication corresponds to composing two permutations, which results in another permutation.

\textbf{2. Associativity}: Matrix multiplication is associative.

\textbf{3. Identity Element}: The identity matrix is a permutation matrix (it corresponds to the identity permutation) and is the identity element in matrix multiplication.

\textbf{4. Inverses}: Every permutation matrix has an inverse, which is also a permutation matrix. This inverse corresponds to the inverse permutation, achieved by reversing the permutation.

This confirms that permutation matrices form a group under matrix multiplication.

\textbf{Counting \(n \times n\) Permutation Matrices}

The number of \(n \times n\) permutation matrices corresponds to the number of different ways to permute \(n\) distinct objects, which is \(n!\) (n factorial). Each permutation matrix is a one-to-one representation of one permutation of the set \(\{1, 2, \ldots, n\}\).

\text{The set of \(n \times n\) permutation matrices forms a group under multiplication, closed under products and inverses, with \(n!\) elements corresponding to each possible permutation of \(\{1, 2, \ldots, n\}\).}


\subsubsection*{Q7: Find by hand the Cholesky decomposition of the matrix}
 \[
   A=
  \left[ {\begin{array}{cc}
   2 & -2 \\
   -2 & 5 \\
  \end{array} } \right]
\]

\textbf{Answer:}

 \[
   A=
  \left[ {\begin{array}{cc}
   a_{11} & a_{12} \\
   a_{21} & a_{22} \\
  \end{array} } \right]
\]
To solve $L_{11}$
\[
L_{11} = \sqrt{a_{11}} = \sqrt{2}
\]
To solve $L_{21}$
\[
L_{21} = \frac{a_{21}}{L_{11}} = \frac{-2}{\sqrt{2}}
\]
To solve $L_{22}$
\[
L_{22} = \sqrt{a_{22} - L_{21}l_{21}}
\]

\[
L_{22} = \sqrt{5 - \frac{-2}{\sqrt{2}} \frac{-2}{\sqrt{2}}}
\]

\[
L_{22} = \sqrt{5 - -\frac{\sqrt{2}\sqrt{2}}{\sqrt{2}} -\frac{\sqrt{2}\sqrt{2}}{\sqrt{2}}}
\]
\[
L_{22} = \sqrt{5 - \sqrt{2}\sqrt{2}}
\]
\[
L_{22} = \sqrt{5 - 2}
\]

\[
L_{22} = \sqrt{3}
\]

You can find $L$ and $L^T$ by plugging it into the following equation 

\[
   L=
  \left[ {\begin{array}{cc}
   \sqrt{2} & 0 \\
    \frac{-2}{\sqrt{2}} & \sqrt{3} \\
  \end{array} } \right]
\]
\[
   L^T=
  \left[ {\begin{array}{cc}
   \sqrt{2} & \frac{-2}{\sqrt{2}} \\
    0 & \sqrt{3} \\
  \end{array} } \right]
\]

\subsubsection*{Q8: Show that the matrices
\[ B = \begin{pmatrix} 1 & 0 & 0 \\ 2 & 0 & 0 \\ 2 & 3 & 2 \end{pmatrix}, \quad C = \begin{pmatrix} 1 & 0 & 0 \\ 2 & 0 & 0 \\ 2 & 0 & \sqrt{13} \end{pmatrix} \]
\text{are both valid Cholesky-like decompositions of the positive semidefinite matrix}}
\[ A = \begin{pmatrix} 1 & 2 & 2 \\ 2 & 4 & 4 \\ 2 & 4 & 17 \end{pmatrix}. \]

To prove that matrices \(B\) and \(C\) are valid Cholesky-like decompositions of matrix \(A\), we need to show that \(B \cdot B^T = A\) and \(C \cdot C^T = A\). 

\textbf{Matrix \(B\)}

Given:
\[ B = \begin{pmatrix} 1 & 0 & 0 \\ 2 & 0 & 0 \\ 2 & 3 & 2 \end{pmatrix} \]

Calculate \(B \cdot B^T\):
\[ B^T = \begin{pmatrix} 1 & 2 & 2 \\ 0 & 0 & 3 \\ 0 & 0 & 2 \end{pmatrix} \]

Multiplying \(B\) by \(B^T\):
\[ B \cdot B^T = \begin{pmatrix} 1 & 2 & 2 \\ 2 & 4 & 4 \\ 2 & 4 & 17 \end{pmatrix} \]

Steps:
\begin{itemize}
    \item 1. For element (1,1): \(1 \times 1 + 0 \times 0 + 0 \times 0 = 1\)
    \item 2. For element (1,2): \(1 \times 2 + 0 \times 0 + 0 \times 0 = 2\) (and symmetrically for (2,1))
    \item 3. For element (1,3): \(1 \times 2 + 0 \times 3 + 0 \times 2 = 2\) (and symmetrically for (3,1))
    \item 4. For element (2,2): \(2 \times 2 + 0 \times 0 + 0 \times 0 = 4\)
    \item 5. For element (2,3): \(2 \times 2 + 0 \times 3 + 0 \times 2 = 4\) (and symmetrically for (3,2))
    \item 6. For element (3,3): \(2 \times 2 + 3 \times 3 + 2 \times 2 = 17\)
\end{itemize}

Thus, \(B \cdot B^T = A\), confirming \(B\) is a valid decomposition.

\textbf{Matrix \(C\)}

Given:
\[ C = \begin{pmatrix} 1 & 0 & 0 \\ 2 & 0 & 0 \\ 2 & 0 & \sqrt{13} \end{pmatrix} \]

Calculate \(C \cdot C^T\):
\[ C^T = \begin{pmatrix} 1 & 2 & 2 \\ 0 & 0 & 0 \\ 0 & 0 & \sqrt{13} \end{pmatrix} \]

Multiplying \(C\) by \(C^T\):
\[ C \cdot C^T = \begin{pmatrix} 1 & 2 & 2 \\ 2 & 4 & 4 \\ 2 & 4 & 17 \end{pmatrix} \]

Steps:
\begin{itemize}
    \item 1. For element (1,1): \(1 \times 1 + 0 \times 0 + 0 \times 0 = 1\)
    \item 2. For element (1,2): \(1 \times 2 + 0 \times 0 + 0 \times 0 = 2\) (and symmetrically for (2,1))
    \item 3. For element (1,3): \(1 \times 2 + 0 \times 0 + 0 \times \sqrt{13} = 2\) (and symmetrically for (3,1))
    \item 4. For element (2,2): \(2 \times 2 + 0 \times 0 + 0 \times 0 = 4\)
    \item 5. For element (2,3): \(2 \times 2 + 0 \times 0 + 0 \times \sqrt{13} = 4\) (and symmetrically for (3,2))
    \item 6. For element (3,3): \(2 \times 2 + 0 \times 0 + \sqrt{13} \times \sqrt{13} = 17\)
\end{itemize}

Thus, \(C \cdot C^T = A\), confirming \(C\) is also a valid decomposition.

\subsubsection*{Suppose that the matrix \( A = (a_{ij}) \) is banded in the sense that \( a_{ij} = 0 \) when \( |i - j| > d \). Prove that the Cholesky decomposition \( L = (l_{ij}) \) also satisfies the band condition \( l_{ij} = 0 \) when \( |i - j| > d \).}

\textbf{Proof: Cholesky Decomposition of a Banded Matrix}

We need to demonstrate that if \(A\) is a banded matrix with bandwidth \(d\) (i.e., \(a_{ij} = 0\) when \(|i - j| > d\)), then its Cholesky decomposition \(L\) will also satisfy this band condition.

\textbf{Step 1: Understanding Cholesky Decomposition}

Cholesky decomposition of a positive definite matrix \(A\) produces a lower triangular matrix \(L\) such that:
\[ A = LL^T \]
Each element of \(A\) can be written in terms of elements of \(L\) as:
\[ a_{ij} = \sum_{k=1}^{\min(i,j)} l_{ik}l_{jk} \]

\textbf{Step 2: Zero Conditions in \(A\)}

Given \(a_{ij} = 0\) for \(|i - j| > d\), this impacts the computation of \(L\) as follows:
- For \(i > j + d\), \(a_{ij} = 0\) and thus:
\[ \sum_{k=1}^j l_{ik}l_{jk} = 0 \]

\textbf{Step 3: Computation of \(L\) and Bandwidth}

We proceed by induction:

\textbf{Base Case:} For \(i = 1\), \(l_{11}\) is determined by \(a_{11}\) and does not depend on any zero conditions outside the bandwidth since there are no such elements.

\textbf{Inductive Step: }Assume that for all \(k < i\), \(l_{kj} = 0\) when \(|k - j| > d\). Now consider \(l_{ij}\) for \(i > j + d\):
  \begin{itemize}
      \item - From the definition of \(L\), \(l_{ij}\) contributes to \(a_{ij}\) only if \(l_{ij} \neq 0\) and \(j \leq i\). However, if \(i > j + d\), we know \(a_{ij} = 0\).
      \item - \(l_{ij}\) is calculated using the formula derived from setting \(a_{ij}\) equal to the sum of products of terms from \(L\), which includes the condition:
      \item \[ l_{ij} = \frac{1}{l_{jj}} \left(a_{ij} - \sum_{k=1}^{j-1} l_{ik}l_{jk}\right) \]
      \item Given that \(l_{ik}\) and \(l_{jk}\) for \(k < j\) adhere to the band condition by inductive hypothesis, and \(a_{ij} = 0\) for \(i > j + d\), all terms in the sum are zero, thereby enforcing \(l_{ij} = 0\) for \(i > j + d\).
  \end{itemize}

\subsubsection*{Q11: Show that inversion of an arbitrary square matrix \(B\) can be reduced to inversion of a positive definite matrix via the identity \(B^{-1} = B^* (B B^*)^{-1}\). Since multiplication of two \(n \times n\) matrices has computational complexity \(O(n^3)\), matrix inversion also has computational complexity \(O(n^3)\). For the record, this is definitely not the preferred method of matrix inversion.}

\textbf{Proof: Inversion of an Arbitrary Square Matrix Using Positive Definite Matrix}

\textbf{The Identity}

The identity to prove is:
\[ B^{-1} = B^* (BB^*)^{-1} \]
where \( B^* \) is the conjugate transpose of \( B \), and \( B \) is any arbitrary square matrix.

\textbf{Step 1: Verification of the Identity}

To verify the correctness of the identity, multiply \( B \) by \( B^{-1} \) as defined:
\[ B B^{-1} = B [B^* (BB^*)^{-1}] \]
Using associativity of matrix multiplication:
\[ = (BB^*)(BB^*)^{-1} \]
Since \( (BB^*) (BB^*)^{-1} = I \), where \( I \) is the identity matrix, this shows:
\[ B B^{-1} = I \]
Thus, \( B^{-1} = B^* (BB^*)^{-1} \) correctly computes the inverse of \( B \).

\textbf{Step 2: Positive Definiteness of \( BB^* \)}

We must show that \( BB^* \) is a positive definite matrix:
\begin{itemize}
    \item 1. Hermitian: \( BB^* \) is Hermitian because \( (BB^*)^* = (B^*)^*B^* = BB^* \).
    \item 2. Positive Semi-Definite: For any non-zero vector \( x \), \( x^*BB^*x \) is non-negative. Specifically,
     \[ x^*BB^*x = (B^*x)^*(B^*x) = \|B^*x\|^2 \geq 0 \]
     Moreover, if \( B \) is invertible, \( B^*x \neq 0 \) for all non-zero \( x \), ensuring \( \|B^*x\|^2 > 0 \), making \( BB^* \) positive definite.
\end{itemize}

\textbf{Step 3: Computational Complexity}

\begin{itemize}
    \item 1. Matrix Multiplication: The multiplication \( BB^* \) involves \( O(n^3) \) operations for an \( n \times n \) matrix.
    \item 2. Matrix Inversion: Inverting \( BB^* \), which is a positive definite matrix, also requires \( O(n^3) \) operations.
\end{itemize}

\subsubsection*{Q13: Find the QR Decomposition of the matrix}

\[
  X=
  \left[ {\begin{array}{ccc}
   1 & 3 & 3\\
   1 & 3 & 1\\
   1 & 1 & 5\\
   1 & 1 & 3\\
  \end{array} } \right]
\]

\textbf{Answer:}\\
To solve $\mu_{1}$
\[
\mu_{1} = a_{1} \textbf{(1,1,1,1)}
\]
To solve $e_{1}$
\[
e_{1} = \frac{\mu_{1}}{||\mu_{1}||} = \frac{1}{\sqrt{4}} (1,1,1,1) 
\]
\[
e_{1}= (\frac{1}{2}, \frac{1}{2}, \frac{1}{2}, \frac{1}{2})
\]
To solve $\mu_{2}$
\[
\mu_2 = a_2 - (a_2 \cdot e_1)e_1, a_2 = (3,3,1,1)
\]
\[
\mu_2 = (3,3,1,1) - 4(\frac{1}{2}, \frac{1}{2}, \frac{1}{2}, \frac{1}{2})
\]
\[
\mu_2 = (3,3,1,1) - (2,2,2,2)
\]
\[
\mu_2 = \textbf{(1,1,-1,-1)}
\]
To solve $e_{2}$
\[
e_2 = \frac{\mu_2}{||\mu_2||} = \frac{1}{\sqrt{4}} = \frac{1}{2}(1,1,-1,-1)
\]
\[
e_2 = (\frac{1}{2}, \frac{1}{2}, -\frac{1}{2},- \frac{1}{2})
\]
To solve $\mu_{3}$
\[
\mu_3 = a_3 - (a_3 \cdot e_1)e_1 - (a_3 \cdot e_2)e_2
\]
\[
\mu_3 = (3,1,5,3) - 6(\frac{1}{2},\frac{1}{2},\frac{1}{2},\frac{1}{2}) - (-2)(\frac{1}{2}, \frac{1}{2}, -\frac{1}{2},- \frac{1}{2})
\]
\[
\mu_3 = (3,1,5,3) - (3,3,3,3) + (1,1,-1,-1)
\]
\[
\mu_3 = (1 -1, 1, -1)
\]
To solve $e_{3}$
\[
e_3 = \frac{\mu_3}{||\mu_3||} = \frac{1}{\sqrt{4}} = \frac{1}{2}(1, -1, 1, -1)
\]
\[
e_3 = (\frac{1}{2}, - \frac{1}{2}, \frac{1}{2}, -\frac{1}{2})
\]
We can find Q using the following formula
\[
Q=[e_1^T | e_2^T | e_3^T]
\]
\[
  Q=
  \left[ {\begin{array}{ccc}
   \frac{1}{2} & \frac{1}{2} & \frac{1}{2}\\
   \frac{1}{2} & \frac{1}{2} & -\frac{1}{2}\\
   \frac{1}{2} & -\frac{1}{2} & \frac{1}{2}\\
   \frac{1}{2} & -\frac{1}{2} & -\frac{1}{2}\\
  \end{array} } \right]
\]
Using the formula 
\[
  R=
  \left[ {\begin{array}{ccc}
   a_1 \cdot e_1 & a_2 \cdot e_1 & a_3 \cdot e_1\\
   0 & a_2 \cdot e_2 & a_3 \cdot e_2\\
   0 & 0 & a_3 \cdot e_3\\
  \end{array} } \right]
\]

And what we know
\[
a_{1} = (1,1,1,1)
\]
\[
a_{2} = (3,3,1,1)
\]
\[
a_{3} = (3,1,5,3)
\]
We can find R

\[
  R=
  \left[ {\begin{array}{ccc}
   2 & 4 & 6\\
   0 & 2 & -2\\
   0 & 0 & 2 \\
  \end{array} } \right]
\]

\subsubsection*{Q14: In linear regression find \(\hat{\beta}\), the predicted values \(\hat{y}\), the residual vector \(r\), and the residual sum of squares \(\|r\|^2\) in terms of the extended QR decomposition
\[
(X, y) = (Q, q) \left(\begin{array}{cc}
R & r \\
0 & d
\end{array}\right).
\]}

To solve the problem, let's start by analyzing the extended QR decomposition given:

\[ (X, y) = (Q, q) \left(\begin{array}{cc} R & r \\ 0 & d \end{array}\right) \]

This can be expanded as:

\[ X = QR \quad \text{and} \quad y = Qr + qd \]

\textbf{Step 1:} Finding \(\hat{\beta}\)

The estimate of \(\beta\) in linear regression using QR decomposition typically involves solving the equation:

\[ R\hat{\beta} = Q^T y \]

Given that \( X = QR \), this can be interpreted as solving the upper triangular system \( R \) from the decomposition of \( X \) against the projection of \( y \) onto the column space of \( Q \):

1. Calculate \( Q^T y \), which is the projection of \( y \) onto the subspace spanned by the columns of \( Q \), leading to:

\[ Q^T y = Q^T (Qr + qd) = Q^T Q r + Q^T q d \]

Since \( Q \) is orthogonal, \( Q^T Q = I \), and assuming \( q \) is orthogonal to \( Q \), \( Q^T q = 0 \). Thus, we simplify this to:

\[ Q^T y = r \]

2. Solving for \(\hat{\beta}\):

\[ R\hat{\beta} = r \]

\(\hat{\beta}\) can be found by solving this system using back substitution since \( R \) is an upper triangular matrix.

\textbf{Step 2:} Finding the Predicted Values \(\hat{y}\)

The predicted values \(\hat{y}\) are given by:

\[ \hat{y} = X\hat{\beta} = QR\hat{\beta} \]

Since we know \(\hat{\beta}\), substituting it in yields:

\[ \hat{y} = QR(R^{-1}r) = Qr \]

\textbf{Step 3: }Finding the Residual Vector \(r\)

In regression, the residual vector \(r\) (different from \(r\) in QR decomposition) is:

\[ r = y - \hat{y} \]

Substituting the known values:

\[ r = (Qr + qd) - Qr = qd \]

\textbf{Step 4:} Finding the Residual Sum of Squares \(\|r\|^2\)

The residual sum of squares \(\|r\|^2\) is:

\[ \|r\|^2 = \|qd\|^2 \]

Since \( q \) is orthogonal and normalized:

\[ \|r\|^2 = d^2 \]

\textbf{Conclusion}

We have derived expressions for \(\hat{\beta}\), \(\hat{y}\), the residual vector \(r\), and the residual sum of squares using the extended QR decomposition:

\begin{itemize}
    \item \(\hat{\beta} = R^{-1}r\)
    \item \(\hat{y} = Qr\)
    \item \(r = qd\)
    \item \(\|r\|^2 = d^2\)
\end{itemize}

This demonstrates how the QR decomposition of \(X\) and \(y\) can be utilized to decompose the regression problem into manageable components, using the orthogonality and triangular properties of \(Q\) and \(R\), respectively.

%q15
\subsubsection*{Q15: If \(X = QR\) is the QR decomposition of \(X\), then show that the projection matrix
\[ X(X^* X)^{-1}X^* = QQ^*. \]
\text{Also show that} \(\det(X) = \det(R)\) \text{when \(X\) is square and in general that}
\[ \det(X^* X) = (\det(R))^2. \]}



\textbf{Part 1: Showing that \(X(X^* X)^{-1}X^* = QQ^*\)}

\textbf{1. Starting from the QR Decomposition:} Given \(X = QR\), where \(Q\) is an orthogonal matrix (\(Q^* Q = I\)) and \(R\) is an upper triangular matrix.

\textbf{2. Calculate \(X^* X\):}
   \[
   X^* X = (QR)^* QR = R^* Q^* QR
   \]
   Since \(Q\) is orthogonal, \(Q^* Q = I\), thus:
   \[
   X^* X = R^* R
   \]

\textbf{3. Compute \((X^* X)^{-1}\):}
   \[
   (X^* X)^{-1} = (R^* R)^{-1} = R^{-1} (R^*)^{-1}
   \]
   Note: This step assumes \(R\) is invertible, which is a reasonable assumption in the context of the QR decomposition when \(X\) has full rank.

\textbf{4. Expression for \(X(X^* X)^{-1}X^*\):}
   \[
   X(X^* X)^{-1}X^* = QR (R^* R)^{-1} R^* Q^* = QR R^{-1} (R^*)^{-1} R^* Q^*
   \]
   Since \(RR^{-1} = I\) and \(R^*(R^*)^{-1} = I\), we get:
   \[
   QR I I Q^* = QQ^*
   \]

\textbf{Part 2: Showing that \(\det(X) = \det(R)\) for square \(X\)}

\textbf{1. Determinant of \(X\):}
   \[
   \det(X) = \det(QR) = \det(Q) \det(R)
   \]
   Since \(Q\) is orthogonal, its determinant is either \(+1\) or \(-1\), thus:
   \[
   \det(X) = \pm \det(R)
   \]
   For an orthogonal matrix \(Q\), if its determinant is \(+1\) or \(-1\), this affects only the sign of the determinant of \(X\), not its absolute value, which aligns with \(\det(R)\) under typical conditions where orientation is preserved (\(\det(Q) = 1\)).

\textbf{Part 3: Showing that \(\det(X^* X) = (\det(R))^2\)}

\textbf{Using previously found \(X^* X = R^* R\):}
   \[
   \det(X^* X) = \det(R^* R) = \det(R^*) \det(R)
   \]
   Since \(R\) is upper triangular, so is \(R^*\), and:
   \[
   \det(R^*) = \det(R)
   \]
   Therefore:
   \[
   \det(X^* X) = (\det(R))^2
   \]

This completes the proof of the given statements in the problem.

% q16
\subsubsection*{Q16: Let \(v_1, \ldots, v_n\) be \(n\) conjugate vectors for the \(n \times n\) positive definite matrix \(A\). Describe how you can use the expansion \(x = \sum_{i=1}^n c_i v_i\) to solve the linear equation \(Ax = b\).}

To solve the linear equation \(Ax = b\) using the expansion \(x = \sum_{i=1}^n c_i v_i\), where \(v_1, \ldots, v_n\) are \(n\) conjugate vectors with respect to the positive definite matrix \(A\), we follow these steps:

\textbf{Step 1: Define the expansion of \(x\)}
Given \(x = \sum_{i=1}^n c_i v_i\), where \(c_i\) are coefficients to be determined.

\textbf{Step 2: Utilize the conjugacy of vectors}
The vectors \(v_1, \ldots, v_n\) are A-conjugate, meaning \(v_i^T A v_j = 0\) for all \(i \neq j\). This property will be essential in simplifying the system of equations derived from substituting \(x\) in \(Ax = b\).

\textbf{Step 3: Substitute \(x\) into the equation \(Ax = b\)}
\[
Ax = A\left(\sum_{i=1}^n c_i v_i\right) = \sum_{i=1}^n c_i Av_i
\]
Since the \(v_i\) are conjugate vectors, the matrix \(A\) acting on any \(v_i\) does not introduce terms involving any \(v_j\) for \(j \neq i\).

\textbf{Step 4: Formulate the system of equations}
To find \(c_i\), project the equation \(Ax = b\) onto each \(v_i\):
\[
v_i^T Ax = v_i^T b \quad \text{for each } i
\]
Substitute \(Ax\) from step 3:
\[
v_i^T \left(\sum_{j=1}^n c_j Av_j\right) = v_i^T b
\]
Using the conjugacy, \(v_i^T Av_j = 0\) for \(i \neq j\), and only the term where \(i = j\) remains:
\[
v_i^T (c_i Av_i) = v_i^T b \quad \Rightarrow \quad c_i v_i^T Av_i = v_i^T b
\]
Thus, solving for \(c_i\):
\[
c_i = \frac{v_i^T b}{v_i^T Av_i}
\]

\textbf{Step 5: Construct \(x\) from the coefficients \(c_i\)
With \(c_i\) found, construct \(x\):}
\[
x = \sum_{i=1}^n \frac{v_i^T b}{v_i^T Av_i} v_i
\]

\textbf{Conclusion}
This method effectively utilizes the orthogonality of the conjugate vectors under transformation by \(A\) to simplify the solution process. Each coefficient \(c_i\) is determined independently of the others, making the calculation straightforward and parallelizable. This is particularly useful in numerical methods such as the Conjugate Gradient method, widely used for solving large systems of linear equations where \(A\) is symmetric and positive definite.

%q17
\subsubsection*{Q17: Suppose that \(A\) is an \(n \times n\) positive definite matrix and that the nontrivial vectors \(u_1, \ldots, u_n\) satisfy
\[ u_i^* A u_j = 0 \quad \text{and} \quad u_i^* u_j = 0 \]
for all \(i \neq j\). Demonstrate that the \(u_i\) are eigenvectors of \(A\).}

%q18
\subsubsection*{Q18: Suppose that the \(n \times n\) symmetric matrix \(A\) satisfies \(v^* A v \neq 0\) for all \(v \neq 0\) and that \(\{u_1, \ldots, u_n\}\) is a basis of \(\mathbb{R}^n\). If one defines \(v_1 = u_1\) and inductively
\[ v_k = u_k - \sum_{j=1}^{k-1} \frac{u_k^* A v_j}{v_j^* A v_j} v_j \]
for \(k = 2, \ldots, n\), then show that the vectors \(v_1, \ldots, v_n\) are conjugate and provide a basis of \({R}^n\). Note that \(A\) need not be positive definite.}

\newpage
\section*{Chapter 6: Newtons Method}
\subsubsection*{Q2: Write a Julia program to solve Lambert’s equation \(we^w = x\) by Newton’s method for \(x > 0\). Prove that the iterates are defined by
\[ w_{n+1} = w_n + \frac{\frac{x}{w_n e^{w_n}}}{w_n + 1}. \]
Make the argument that \(w_{n+1} > w_n\) when \(w_n e^{w_n} < x\) and that \(w_{n+1} < w_n\) when \(w_n e^{w_n} > x\).}

\subsubsection*{Q3: In solving Lambert’s equation \(we^w = x\) by Newton’s method, one must seed the algorithm with an initial guess. Argue that \(w_0 = \ln x - \ln(\ln x)\) is good for moderate to large \(x\). Show that it is exact for \(x = e\). For small \(x\), argue that the guess \(w_0 = \frac{x}{1 + cx}\) is good. What value of the constant \(c\) solves Lambert’s equation when \(x = e\)? Plot these approximations against the solution curve of Lambert’s equation.}

\subsubsection*{Q4: Halley’s method for finding a root of the equation \(f(x) = 0\) approximates \(f(x)\) around \(x_n\) by the quadratic
\[ q(x) = f(x_n) + f'(x_n)(x - x_n) + \frac{1}{2} f''(x_n)(x - x_n)^2. \]
It then rearranges the equation \(q(x) = 0\) in the form
\[ x - x_n = -\frac{f(x_n)}{f'(x_n) + \frac{1}{2} f''(x_n)(x - x_n)} \]
and then approximates \(x - x_n\) on the right-hand side by the Newton increment \(-f(x_n)/f'(x_n)\). Show that these maneuvers yield the Halley update
\[ x_{n+1} = x_n - \frac{2f(x_n)f'(x_n)}{2f'(x_n)^2 - f(x_n)f''(x_n)}. \]
Halley’s method has a cubic rate of convergence. Compare its practical performance to Newton’s method in solving Lambert’s equation of the previous two problems.}

\subsubsection*{Q5: Consider the function
\[ f(x) = \begin{cases} 
0 & \text{if } x = 0 \\
x + x^2 \sin\left(\frac{2}{x}\right) & \text{if } x \neq 0.
\end{cases} \]
Calculate its derivative \(f'(x)\), and argue that Newton’s method tends to be repelled by its root \(x = 0\). This failure occurs despite the fact that \(f(x)\) possesses a bounded derivative in a neighborhood of 0.}

\subsubsection*{Q6: The function \(f(x) = x + x^{4/3}\) has \(x = 0\) as a root. Derive the Newton updates}
\[ x_{n+1} = \frac{\frac{1}{3} x_n^{4/3}}{1 + \frac{4}{3} x_n^{1/3}}, \]
\text{and show that}
\[ \lim_{n \to \infty} \frac{x_{n+1}}{x_n^{4/3}} = \frac{1}{3} \]
\text{for \(x_0\) close to 0. This subquadratic rate of convergence occurs because \(f(x)\) is not twice differentiable at 0.}

\subsubsection*{Q7: Characterize the behavior of Newton’s method in minimizing the function \(f(x) = \sqrt{x^2 + 1}\). When the method converges, what is its order of convergence?}

\subsubsection*{Q8: For any positive number \(x\), show that
\[ \log_2 x = m \pm \log_2 (1 + w) = m \pm \frac{\ln(1 + w)}{\ln 2} \]
for an integer \(m\) and a real \(w \in \left[0, \frac{1}{2}\right]\) [177]. (Hints: Write \(x = 2^n + b\) with
\[ 0 \leq b < 2^n. \text{ Then } \log_2 x = n + \log_2 (1 + r) \text{ for } r = \frac{b}{2^n}. \text{ When } r \geq \frac{1}{2}, \text{ write } \]
\[ n + \log_2 (1 + r) = n + 1 - \log_2 \left(1 + \frac{1 - r}{1 + r}\right). \]}

\subsubsection*{Q9: For \(y\) positive the positive root of the equation \(f(x) = \frac{1}{x^2} - y = 0\) is \(\frac{1}{\sqrt{y}}\). Show that the Newton iterates for finding the root are
\[ x_{n+1} = \frac{x_n(3 - yx_n^2)}{2}. \]
Alternatively, \(x = \frac{1}{\sqrt{y}}\) solves the equation \(g(x) = yx^2 - 1 = 0\). Demonstrate that this formulation gives rise to the Newton updates
\[ x_{n+1} = \frac{1}{2} \left( x_n + \frac{1}{yx_n} \right). \]
The first scheme involves no reciprocals, but the second scheme has better convergence guarantees. Show that the second scheme satisfies \(x_{n+1} \geq \frac{1}{\sqrt{y}}\) regardless of the value of \(x_n > 0\). Also show that \(x_{n+1} \leq x_n\) whenever \(x_n \geq \frac{1}{\sqrt{y}}\). Hence, global convergence is assured.}

\subsubsection*{Q10: The binomial theorem states that for \(r\) real and \(|x| < 1\)
\[ (1 + x)^r = \sum_{k=0}^{\infty} \binom{r}{k} x^k = \sum_{k=0}^{\infty} \frac{r(r-1) \cdots (r-k+1)}{k!} x^k. \]
\text{One can adapt this recipe to calculate matrix roots by defining}
\[ (I + X)^r = \sum_{k=0}^{\infty} \binom{r}{k} X^k \]
for square matrices \(X = (x_{ij})\) with norm \(\|X\| < 1\) [81]. Here the norm is generic except for the requirement that \(\|AB\| \leq \|A\| \cdot \|B\|\). The most convenient choice is the Frobenius norm \(\|X\|_F^2\). One can extend the series method to matrices \(Y\) with larger norms by writing \(Y^r = c^r (I + X)^r\), where \(X = \frac{1}{c}Y - I\) for some constant \(c\). Show that under the Frobenius norm the optimal choice of \(c\) satisfies
\[ c = \frac{\|Y\|_F^2}{\operatorname{tr}(Y)} \quad \text{and} \quad \|X\|_F^2 = \|I\|_F^2 - \frac{\operatorname{tr}(Y)^2}{\|Y\|_F^2}. \]
\text{When \(\operatorname{tr}(Y)\) is negative, the factor \(c^r\) may be complex.}}

\subsubsection*{Q11: For the weighted least squares criterion
\[ f(\beta) = \frac{1}{2} \sum_{i=1}^n w_i \left( y_i - \sum_{j=1}^p x_{ij} \beta_j \right)^2, \]
\text{prove that the minimum is achieved when}
\[ \beta = (X^* W X)^{-1} X^* W y, \]
\text{where \(W\) is a diagonal matrix with \(i\)-th diagonal entry \(w_i > 0\).}}

\subsubsection*{Q12: Demonstrate Woodbury’s generalization
\[ (A + UBV^*)^{-1} = A^{-1} - A^{-1}U(B^{-1} + V^* A^{-1} U)^{-1} V^* A^{-1} \]
of the Sherman–Morrison matrix inversion formula for compatible matrices \(A\), \(B\), \(U\), and \(V\).}

\subsubsection*{Q13: To explore whether the performance of Newton’s method can be improved by a linear change of variables, consider solving the two problems \(f(x) = 0\) and \(Af(Bx) = 0\), where \(A\) and \(B\) are invertible matrices of the right dimensions. Show that the two methods lead to basically the same iterates when started at \(x_0\) and \(B^{-1} x_0\), respectively.}

\subsubsection*{Q14: To solve the vector-valued equation \(g(x) = 0\), one can minimize the function
\[ f(x) = \frac{1}{2} \|g(x)\|^2. \]
\text{Show that \(f(x)\) has gradient}
\[ \nabla f(x) = dg(x)^* g(x) \]
\text{and second differential}
\[ d^2 f(x) = dg(x)^* dg(x) + d^2 g(x)^* g(x) \approx dg(x)^* dg(x). \]
The approximation to the second differential is good when \(x\) is close to a root. When the number of components of \(g(x)\) equals the dimension of \(x\), prove that the combination of this approximation and Newton’s method of minimization leads to the standard Newton update
\[ x_{n+1} = x_n - dg(x_n)^{-1} g(x_n). \]
Finally, argue in this case that the Newton increment is a descent direction for the objective function \(f(x)\).}

\newpage
\section*{Chapter 7: Linear Programming}

\subsubsection*{Q2: Consider the linear program of maximizing \(x_1 + 2x_2 + 3x_3 + 4x_4 + 5\) subject to the constraints
\[
\begin{aligned}
4x_1 + 3x_2 + 2x_3 + x_4 &\leq 10 \\
x_1 - x_3 + 2x_4 &= 2 \\
x_1 + x_2 + x_3 + x_4 &\geq 1
\end{aligned}
\]
and \(x_1 \geq 0, x_3 \geq 0, x_4 \geq 0\). Put this program into canonical form and solve.}

\subsubsection*{Q3: Convert the problem of minimizing \(|x_1 + x_2 + x_3|\) subject to \(x_1 - x_2 = 5\), \(x_2 - x_3 = 7\), and \(x_1 \geq 0\), and \(x_3 \geq 2\) into a linear program and solve.}

\subsubsection*{Q4: Convert the problem of minimizing \(|x_1| - |x_2|\) subject to \(x_1 + x_2 = 5\), \(2x_1 + 3x_2 - x_3 \leq 0\), and \(x_3 \geq 4\) into a linear program and solve.}

\subsubsection*{Q5: Find an upper bound on the number of basic feasible points of a linear program.}

\subsubsection*{Q6: A set \(C\) is said to be convex if whenever \(\mathbf{u}\) and \(\mathbf{v}\) belong to \(C\), then the entire line segment \([\mathbf{u}, \mathbf{v}] = \{t \mathbf{u} + (1 - t) \mathbf{v} : t \in [0, 1]\}\) belongs to \(C\). Show that the feasible region of a linear program is convex. A set \(C\) is said to be closed if whenever a sequence \(x_n\) from \(C\) converges to a limit \(x\), then \(x\) also belongs to \(C\). Show that the feasible region of a linear program is closed.}

\subsubsection*{Q7: Give an example of a linear program whose feasible region \(R\) is unbounded. Construct an objective \(c^* x\) that is bounded below on \(R\) and an objective \(c^* x\) that is unbounded below on \(R\).}

\subsubsection*{Q8: A point \(\mathbf{x}\) of a convex set \(C\) is called extreme if it cannot be expressed as a nontrivial convex combination \(\mathbf{x} = t\mathbf{y} + (1 - t)\mathbf{z}\) of two distinct points \(\mathbf{y}\) and \(\mathbf{z}\) from \(C\). Here nontrivial means that \(t\) occurs in the open interval \((0, 1)\). Prove that a point \(\mathbf{x}\) in the feasible region \(\{\mathbf{x} : A\mathbf{x} = \mathbf{b}, \; \mathbf{x} \geq 0\}\) is extreme if and only if the columns \(A_B\) of \(A\) associated with its support set \(B = \{i : x_i > 0\}\) are linearly independent.}

\subsubsection*{Q9: The dual function of a linear program is defined by
\[ \mathcal{D}(\lambda, \mu) = \min_x \mathcal{L}(x, \lambda, \mu), \]
\text{where \(\mathcal{L}(x, \lambda, \mu)\) is the Lagrangian (7.1). Prove that the dual equals}
\[ \mathcal{D}(\lambda, \mu) = \begin{cases} 
-\mathbf{b}^* \lambda & \mathbf{A}^* \lambda \leq \mathbf{c} \\
-\infty & \text{otherwise}.
\end{cases} \]}

\subsubsection*{Q10: In the notation of the previous problem, demonstrate the duality result
\[
\max_{\{(\lambda, \mu) : \mu \geq 0\}} \mathcal{D}(\lambda, \mu) = \min_{\{x : A x = b, x \geq 0\}} \mathbf{c}^* x
\]
\text{when the linear program has a solution. (Hints: For \(x\) feasible show that}
\[
\mathcal{D}(\lambda, \mu) \leq \mathcal{L}(x, \lambda, \mu) \leq \mathbf{c}^* x.
\]
\text{At the constrained minimum \(y\) with multipliers \(\hat{\lambda}\) and \(\hat{\mu}\), also show}
\[
\mathbf{c}^* y = \mathcal{L}(y, \hat{\lambda}, \hat{\mu}) = \mathcal{D}(\hat{\lambda}, \hat{\mu}).
\]}

\subsubsection*{Q11: Let \(A\) be a matrix with full row rank and
\[ S = \{x \in \mathbb{R}^p : Ax = b\} \]
\text{be an affine subspace. Show that the closest point to \(y\) in \(S\) is}
\[ P_S(y) = y - A^* (AA^*)^{-1} (Ay - b) \]
by minimizing the function \(f(x) = \frac{1}{2} \|y - x\|^2\) subject to the constraint \(Ax = b\). The matrix \(P = I - A^* (AA^*)^{-1} A\) is an orthogonal projection. Check the properties \(P^2 = P\), \(P^* = P\), and \(Px = x\) for \(x \in S\).}

\subsubsection*{Q12: Design and implement appropriate numerical examples to compare the performance of the revised simplex method and Karmarkar’s algorithm.}

\subsubsection*{Q13: As an alternative to the revised simplex method and Karmarkar’s algorithm, one can derive a path-following algorithm to solve the linear programming problem [30]. Consider minimizing \(\mathbf{c}^* \mathbf{x}\) subject to the standard linear programming constraints \(\mathbf{x} \geq 0\) and \(A \mathbf{x} = \mathbf{b}\). Given an initial feasible point \(\mathbf{x}_0 > 0\), one can devise a differential equation \(\frac{d}{dt} \mathbf{x}(t) = G(\mathbf{x})\) whose solution \(\mathbf{x}(t)\) is likely to converge to the optimal point. Simply take \(G(\mathbf{x}) = D(\mathbf{x}) P(\mathbf{x}) \mathbf{v}(\mathbf{x})\), where \(D(\mathbf{x}) = \operatorname{diag}(\mathbf{x})\) is a diagonal matrix with diagonal entries given by the vector \(\mathbf{x}\) and \(P(\mathbf{x})\) is orthogonal projection onto the null space of \(A D(\mathbf{x})\). (See the previous problem.) The matrix \(D(\mathbf{x})\) slows the trajectory down as it approaches a boundary \(x_i = 0\). The matrix \(P(\mathbf{x})\) ensures that the value of \(A \mathbf{x}(t)\) remains fixed at the constant \(\mathbf{b}\). Check this fact. Show that the choice \(\mathbf{v}(\mathbf{x}) = -P(\mathbf{x}) D(\mathbf{x}) c\) yields \(\frac{d}{dt} \mathbf{c}^* \mathbf{x}(t) \leq 0\). In other words, \(\mathbf{c}^* \mathbf{x}(t)\) is a Liapunov function for the solution path.}

\subsubsection*{Q14: Reduce the path-following algorithm of the previous problem to Julia code. This can be accomplished by adopting Euler’s method for solving the differential equation \(\frac{d}{dt} \mathbf{x}(t) = G(\mathbf{x})\). In Euler’s method we approximate
\[ \mathbf{x}(t + \delta) \approx \mathbf{x}(t) + \delta \frac{d}{dt} \mathbf{x}(t) = \mathbf{x}(t) + \delta G(\mathbf{x}) \]
for \(\delta > 0\) small. Note that Julia has a simple command to compute the pseudoinverse \(M^* (M M^*)^{-1}\) of a matrix \(M\) with full row rank. Your code must start with an \(\mathbf{x}\) having all entries positive and satisfying \(A \mathbf{x} = \mathbf{b}\).}

\newpage
\section*{Chapter 8: Eigenvalues and Eigenvectors}

\subsubsection*{Q1: Describe the behavior of the power method applied to the matrix
\[ 
\mathbf{A} = 
\begin{pmatrix}
0 & 1 & 0 & 0 \\
-1 & 0 & 0 & 0 \\
0 & 0 & 1 & 1 \\
0 & 0 & 0 & 1 
\end{pmatrix}. 
\]
Explain your empirical findings by invoking the \texttt{eigen} command of Julia to reveal \(\mathbf{A}\)'s eigenvalues and eigenvectors.}

\subsubsection*{Q2: Find the eigenvalues and eigenvectors of the matrix
\[
\mathbf{A} = 
\begin{pmatrix}
10 & 7 & 8 & 7 \\
7 & 5 & 6 & 5 \\
8 & 6 & 10 & 9 \\
7 & 5 & 9 & 10 
\end{pmatrix}
\]
of RJ Wilson by divide and conquer and Jacobi’s method.}

\subsubsection*{Q3: Find the eigenvalues and eigenvectors of the rotation matrix
$$
\begin{pmatrix}
\cos \theta & -\sin \theta \\
\sin \theta & \cos \theta
\end{pmatrix}
$$
Note that the eigenvalues are complex conjugates}

First solve for $det(A- \lambda I) = 0$ and find the roots for the given polynomial
$$
det \begin{pmatrix}
\cos \theta - \lambda & -\sin \theta \\
\sin \theta & \cos \theta - \lambda
\end{pmatrix} = (\cos \theta - \lambda)^{2} - (-\sin \theta) = 0 
$$

$$
\lambda ^{2}-2 \lambda \cos \theta + \cos ^{2} \theta + \sin ^{2} \theta = 0
$$

We can now use the trig identity to further simplify the expression $ \cos ^{2} \theta + \sin ^{2} \theta = 1$ to obtain: 

$$
\lambda ^{2}-2 \lambda \cos \theta + 1 = 0
$$

We now plug in this to the quadratic formula to find its roots:

$$
\lambda = \frac{-2 \lambda \cos \theta \pm \sqrt{4 \lambda \cos ^{2} \theta - 4(\lambda ^{2})(1)}}{2 \lambda ^{2}} = \cos \theta \pm \sqrt{\cos ^{2} \theta -1 } 
$$

This expression simplifies further using the relation $\cos \theta \pm i \sin \theta = e^{\pm i \theta}$to obtain the complex conjugate roots:

$$
\lambda = \cos \theta \pm \sqrt{\cos ^{2} \theta -1 } = \cos \theta \pm i \sin \theta = e^{\pm i \theta}.
$$

We have therefore found not real eigenvalues if $\theta \neq 0$ (and every $\pm \pi$). However if $theta = 0, \pi$, we would get real eigenvalues.

Next we find the eigenvectors of the following by plugging in our eigenvalues to the original $A- \lambda I$.

\textbf{Case 1:} First lets solve for the case where $\theta = 0, \pi$.

$$
A - \lambda I
\begin{pmatrix}
 -i \sin \theta & -\sin \theta \\
\sin \theta & -i \sin \theta 
\end{pmatrix} = 
\begin{pmatrix}
 0 & 0 \\
0 & 0 
\end{pmatrix} 
$$

This shows that each nonzero vector of $\mathbb{R}^{2}$ is an eigenvector \\

\textbf{Case 2:} In the case where $ \theta \neq 0, \pi$. We reduce the row 2 by $R_2 \dot i$ and $\frac{1}{\sin \theta} R_2$ and then subtract $R_2 + R_1$ to obtain: 

$$
A - \lambda I
\begin{pmatrix}
 -i \sin \theta & -\sin \theta \\
\sin \theta & -i \sin \theta 
\end{pmatrix} = 
\begin{pmatrix}
 1 & -i \\
1 & -i 
\end{pmatrix} =
\begin{pmatrix}
 -i & 1 \\
0 & 0 
\end{pmatrix}
$$

Thus we obtain eigenvectors   
\begin{align}
    y &= \begin{bmatrix}
           i \\
        1
         \end{bmatrix}t 
\end{align}
for any nonzero scalar $t$. The other eigenvector is its complex conjugate to obtain the eigenvectors of all cases of the rotation matrix: 
\begin{align}
    y &= \begin{bmatrix}
           -i \\
           1
         \end{bmatrix}t 
\end{align}




\subsubsection*{Q4: Find the eigenvalues and eigenvectors of the reflection matrix
$$
\begin{pmatrix}
\cos \theta & \sin \theta \\
\sin \theta & -\cos \theta
\end{pmatrix}
$$}

First solve for $det(A- \lambda I) = 0$ and find the roots for the given polynomial:

$$
det \begin{pmatrix}
\cos \theta - \lambda & \sin \theta \\
\sin \theta & -\cos \theta - \lambda
\end{pmatrix} = (\cos \theta - \lambda) (-\cos \theta - \lambda) - (\sin \theta)^{2} = 0 
$$

You can further simplify the expression again using the $ \cos ^{2} \theta + \sin ^{2} \theta = 1$ relation:
$$
(\cos \theta - \lambda) (-\cos \theta - \lambda) - (-\sin \theta)^{2} = \lambda ^{2} - \cos ^{2} \theta - \sin ^{2} \theta = \lambda ^{2} -1
$$

Our eigenvalues are therefore $\lambda = \pm 1$ \\

\textbf{Case 1: }We now solve for the eigenvectors by plugging in the eigenvalues into the $A- \lambda I$ equation. 

So we begin with the $\lambda=1$ case. We multiply the $R_1$ by $\cos(\theta) +1$ and multiply $sin(\theta)$ to $R_2$ to reduce the matrix

$$
\begin{pmatrix}
\cos \theta -1 & \sin \theta \\
\sin \theta & -\cos \theta - 1
\end{pmatrix} = 
\begin{pmatrix}
(\cos \theta -1)(\cos \theta +1) & \sin \theta (\cos \theta +1)\\
\sin \theta (\sin \theta) & (-\cos \theta -1)(\sin \theta)
\end{pmatrix} 
$$

$$
= \begin{pmatrix}
-\sin^2 \theta & \sin \theta \cos \theta + \sin(\theta))\\
\sin^2 \theta & -\sin \theta\cos \theta -\sin \theta
\end{pmatrix} =
\begin{pmatrix}
\cos \theta -1 & \sin \theta \\
0 & 0
\end{pmatrix} = 
\begin{pmatrix}
1 & \frac{\sin \theta}{\cos \theta -1} \\
0 & 0
\end{pmatrix}
$$

Solving these equations we get the eigenvector for $\lambda =1$ is 

\begin{align}
    y &= \begin{bmatrix}
           \frac{-\sin \theta }{\cos \theta -1} \\
           1
         \end{bmatrix}t 
\end{align}

\textbf{Case 2: }For the $\lambda =-1$ the system becomes
$$
\begin{pmatrix}
\cos \theta - (-1) & \sin \theta \\
\sin \theta & -\cos \theta -  (-1)
\end{pmatrix} 
$$

We then do a similar one to the first eigenvector and solve multiply the $R_1$ by $\cos(\theta) -1$ and multiply $sin(\theta)$ to $R_2$ to reduce the matrix

$$
\begin{pmatrix}
(\cos \theta + 1)(\cos \theta -1) & \sin \theta (\cos \theta - 1)\\
\sin \theta (\sin \theta) & (-\cos \theta +1)(\sin \theta)
\end{pmatrix} 
$$

$$
= \begin{pmatrix}
-\sin^2 \theta & \sin \theta \cos \theta - \sin(\theta))\\
\sin^2 \theta & -\sin \theta\cos \theta +\sin \theta
\end{pmatrix} =
\begin{pmatrix}
\cos \theta +1 & \sin \theta \\
0 & 0
\end{pmatrix} =
\begin{pmatrix}
1 & \frac{\sin \theta}{\cos \theta +1} \\
0 & 0
\end{pmatrix}
$$


Solving these systems we obtain the eigenvector 

\begin{align}
    y &= \begin{bmatrix}
           \frac{-\sin \theta }{\cos \theta +1} \\
           1
         \end{bmatrix}t 
\end{align}

\subsubsection*{Q5: Suppose \(\lambda\) is an eigenvalue of the orthogonal matrix \(\mathbf{O}\) with corresponding eigenvector \(\mathbf{v}\). Show that \(\mathbf{v}\) has real entries only if \(\lambda = \pm 1\).}

\subsubsection*{Q6: A matrix \(\mathbf{A}\) with real entries is said to be skew-symmetric if \(\mathbf{A}^* = -\mathbf{A}\). Show that all eigenvalues of a skew-symmetric matrix are imaginary or 0.}

\subsubsection*{Q7: Continuing the previous problem, suppose that \(\mathbf{A}\) is skew-symmetric. Prove that \(I - \mathbf{A}\) is invertible and that \((I - \mathbf{A})^{-1}(I + \mathbf{A})\) is orthogonal. The latter matrix is called the Cayley transform of \(\mathbf{A}\).}

\subsubsection*{Q8: Consider an \(n \times n\) upper triangular matrix \(U\) with distinct nonzero diagonal entries. Let \(\lambda\) be its \(m\)th diagonal entry, and write
\[
U = \begin{pmatrix}
U_{11} & U_{12} & U_{13} \\
0 & \lambda & U_{23} \\
0 & 0 & U_{33}
\end{pmatrix}
\]
in block form. Verify that \(\lambda\) is an eigenvalue of \(U\) with eigenvector
\[
w = \begin{pmatrix}
v \\
-1 \\
0
\end{pmatrix}, \quad v = (U_{11} - \lambda I_{m-1})^{-1} U_{12},
\]
where \(I_{m-1}\) is the \((m - 1) \times (m - 1)\) identity matrix.}

\subsubsection*{Q9: Suppose the \(m \times m\) symmetric matrix \(A\) has eigenvalues
\[
\lambda_1 < \lambda_2 \le \cdots \le \lambda_{m-1} < \lambda_m.
\]
The iterative scheme \(x_{n+1} = (A - \eta_n I)x_n\) can be used to approximate either \(\lambda_1\) or \(\lambda_m\) [79]. Consider the criterion
\[
\sigma_n = \frac{x_{n+1}^* A x_{n+1}}{x_{n+1}^* x_{n+1}}.
\]
Choosing \(\eta_n\) to maximize \(\sigma_n\) causes \(\lim_{n \to \infty} \sigma_n = \lambda_m\), while choosing \(\eta_n\) to minimize \(\sigma_n\) causes \(\lim_{n \to \infty} \sigma_n = \lambda_1\). If \(\tau_k = x_n^* A^k x_n\), then show that the extrema of \(\sigma_n\) as a function of \(\eta\) are given by the roots of the quadratic equation}
\[
0 = \det \begin{pmatrix}
1 & \eta & \eta^2 \\
\tau_0 & \tau_1 & \tau_2 \\
\tau_1 & \tau_2 & \tau_3
\end{pmatrix}.
\]

\subsubsection*{Q10: Apply the algorithm of the previous problem to find the largest and smallest eigenvalues of the matrix in problem (2).}

\subsubsection*{Q11: Let \(A\) be a symmetric matrix. Show that the Rayleigh quotient \(R(x) = \frac{x^* A x}{x^* x}\) has gradient
\[
\frac{2[Ax - R(x)x]}{x^* x}.
\]
Argue that the eigenvalues and eigenvectors of \(A\) are the stationary values and stationary points, respectively, of \(R(x)\).}

\subsubsection*{Q12: Denote the smallest and largest eigenvalues of an \(m \times m\) symmetric matrix \(C\) by \(\lambda_1[C]\) and \(\lambda_m[C]\). For any two \(m \times m\) symmetric matrices \(A\) and \(B\) and any \(\alpha \in [0, 1]\), demonstrate that
\[
\lambda_1[\alpha A + (1 - \alpha)B] \geq \alpha \lambda_1[A] + (1 - \alpha) \lambda_1[B]
\]
\[
\lambda_m[\alpha A + (1 - \alpha)B] \leq \alpha \lambda_m[A] + (1 - \alpha) \lambda_m[B].
\]
}

\subsubsection*{Q13: Given the assumptions of the previous problem, show that the smallest and largest eigenvalues satisfy
\[
\lambda_1[A + B] \geq \lambda_1[A] + \lambda_1[B]
\]
\[
\lambda_m[A + B] \leq \lambda_m[A] + \lambda_m[B].
\]
}

\subsubsection*{Q14: For symmetric matrices \(A\) and \(B\), define \(A \triangleright 0\) to mean that \(A\) is positive semidefinite and \(A \triangleright B\) to mean that \(A - B \triangleright 0\). Show that \(A \triangleright B\) and \(B \triangleright C\) imply \(A \triangleright C\). Also show that \(A \triangleright B\) and \(B \triangleright A\) imply \(A = B\). Thus, \(\triangleright\) induces a partial order on the set of symmetric matrices.}

\subsubsection*{Q15: In the notation of the previous problem, show that two positive definite matrices \(A\) and \(B\) satisfy \(A \triangleright B\) if and only if they satisfy \(B^{-1} \triangleright A^{-1}\). If \(A \triangleright B\), then prove that \(\det A \geq \det B\) and \(\operatorname{tr} A \geq \operatorname{tr} B\).}

\subsubsection*{Q16: Suppose the symmetric matrices \(A\) and \(B\) satisfy \(A \triangleright B\) in the notation of the previous two problems. If in addition \(\operatorname{tr} (A) = \operatorname{tr} (B)\), then demonstrate that \(A = B\). (Hint: Consider the matrix \(C = A - B\).)}

\subsubsection*{Q17: Let \(A\) and \(B\) be positive semidefinite matrices of the same dimension. Prove that the matrix \(aA + bB\) is positive semidefinite for every pair of nonnegative scalars \(a\) and \(b\). Thus, the set of positive semidefinite matrices is a convex cone.}

\subsubsection*{Q18: One of the simplest ways of showing that a symmetric matrix is positive semidefinite is to show that it is the covariance matrix of a random vector. Use this insight to prove that if the symmetric matrices \(A = (a_{ij})\) and \(B = (b_{ij})\) are positive semidefinite, then the matrix \(C = (c_{ij})\) with entries \(c_{ij} = a_{ij} b_{ij}\) is also positive semidefinite [138]. (Hint: Take independent random vectors \(\mathbf{x}\) and \(\mathbf{y}\) with covariance matrices \(A\) and \(B\) and form the random vector \(\mathbf{z}\) with components \(z_i = x_i y_i\).)}

\subsubsection*{Q19: Continuing the previous problem, suppose that the \(n \times n\) symmetric matrices \(A\) and \(B\) have entries \(a_{ij} = i(n - j + 1)\) and \(b_{ij} = \sum_{k=1}^i \sigma_k^2\) for \(j \geq i\) and all \(\sigma_k^2 \geq 0\). Show that \(A\) and \(B\) are positive semidefinite [138]. (Hint: For \(A\), consider the order statistics from a random sample of the uniform distribution on \([0, 1]\).)}

\newpage
\section*{Chapter 9: MM Algorithms}
\subsubsection*{Q1: In a majorization-minimization algorithm with objective function \(f(x)\) and surrogate function \(g(x \mid x_n)\), show that the sequence \(g(x_{n+1} \mid x_n)\) decreases.}

\subsubsection*{Q2: In an MM algorithm having a differentiable objective \(f(x)\), a differentiable surrogate \(g(x \mid y)\), and no constraints, prove that the gradient identity \(\nabla f(x) = \nabla g(x \mid x)\) when \(y = x\). Here the gradient of \(g(x \mid y)\) is taken with respect to its left argument \(x\).}

\subsubsection*{Q3: In the previous problem assume that \(f(x)\) and \(g(x \mid y)\) are twice differentiable in \(x\) for each \(y\). Demonstrate that the difference matrix \(d^2 g(x \mid x) - d^2 f(x)\) is positive semidefinite.}

\subsubsection*{Q4: Prove that the function \(\ln \Gamma(t)\) is convex. (Hint: Express \( \frac{d^2}{dt^2} \ln \Gamma(t)\) as the variance of \(\ln X\) for a certain random variable \(X\).)}

\subsubsection*{Q5: Derive the balanced ANOVA estimates (9.8) by the method of Lagrange multipliers.}

\subsubsection*{Q6 Prove the majorization
\[
(x + y - z)^2 \leq -(x_n + y_n - z_n)^2 + 2(x_n + y_n - z_n)(x + y - z) + 3[(x - x_n)^2 + (y - y_n)^2 + (z - z_n)^2]
\]
separating the variables \(x\), \(y\), and \(z\). In Example 9.5 this would facilitate penalizing parameter curvature rather than changes in parameter values.}

\subsubsection*{Q7 Find a quadratic upper bound majorizing the function \( e^{-x^2} \) around the point \( x_n \).}

\subsubsection*{Q8 For the function \( f(x) = \ln(1 + e^x) \), derive the majorization
\[
f(x) \leq f(x_n) + f'(x_n)(x - x_n) + \frac{1}{8} (x - x_n)^2
\]
by the quadratic upper bound principle.}

\subsubsection*{Q9 Demonstrate the majorizations}
\[
xy \leq \frac{1}{2} (x^2 + y^2) + \frac{1}{2} (x_n - y_n)^2 - (x_n - y_n)(x - y)
\]
\[
-xy \leq \frac{1}{2} (x^2 + y^2) + \frac{1}{2} (x_n + y_n)^2 - (x_n + y_n)(x + y).
\]

\subsubsection*{Q10 Based on problem (9), consider minimizing Rosenbrock’s function
\[
f(x) = 100(x_1^2 - x_2)^2 + (x_1 - 1)^2.
\]
Show that up to an irrelevant constant \(f(x)\) is majorized by the sum of the two functions
\[
g_1(x_1 | x_n) = 200x_1^4 - [200(x_n^2 + x_{n2}) - 1]x_1^2 - 2x_1
\]
\[
g_2(x_2 | x_n) = 200x_2^2 - 200(x_{n1}^2 + x_{n2})x_2.
\]
Hence, to implement the corresponding MM algorithm, one must minimize a quartic in \(x_1\) and a quadratic in \(x_2\) at each iteration. Program the MM algorithm, and check whether it converges to the global minimum of \(f(x)\) at \(x = 1\).}

\subsubsection*{Q11 Verify the majorization \(- \ln x \leq - \ln x_n + \frac{x_n}{x} - 1\) for \(x\) and \(x_n\) positive. Use this to design an MM algorithm for minimizing the convex function \(f(x) = ax - \ln x\) for \(a > 0\). Why do the iterates converge to \(a^{-1}\)?}

\subsubsection*{Q12 A number \(\mu_q\) is said to be a \(q\) quantile of the \(m\) numbers \(x_1, \ldots, x_m\) if it satisfies
\[
\frac{1}{m} \sum_{x_i \leq \mu_q} 1 \geq q \quad \text{and} \quad \frac{1}{m} \sum_{x_i \geq \mu_q} 1 \geq 1 - q.
\]
If we define
\[
\rho_q(r) = \begin{cases} 
qr & r \geq 0 \\ 
-(1 - q)r & r < 0 
\end{cases},
\]
then it turns out that \(\mu\) is a \(q\) quantile if and only if \(\mu\) minimizes the function
\[
f_q(\mu) = \sum_{i=1}^m \rho_q(x_i - \mu).
\]
Medians correspond to the case \(q = 1/2\). Show that \(\rho_q(r)\) is majorized by the quadratic
\[
\zeta_q(r \mid r_n) = \frac{1}{4} \left[ \frac{r^2}{|r_n|} + (4q - 2)r + |r_n| \right].
\]
Deduce from this majorization the MM algorithm
\[
\mu_{n+1} = \frac{m(2q - 1) + \sum_{i=1}^m w_{ni} x_i}{\sum_{i=1}^m w_{ni}}
\]
\[
w_{ni} = \frac{1}{|x_i - \mu_n|}
\]
for finding a \(q\) quantile. This interesting algorithm involves no sorting, only arithmetic operations.}

\subsubsection*{Q15 A random variable \(X\) concentrated on the nonnegative integers is said to have a power series distribution if
\[
\Pr(X = k) = \frac{c_k \theta^k}{q(\theta)}.
\]
In equation (9.13), \(\theta\) is a positive parameter, the coefficients \(c_k\) are nonnegative, and \(q(\theta) = \sum_{k=0}^\infty c_k \theta^k\) is the appropriate normalizing constant. Examples include the binomial, negative binomial, Poisson, and logarithmic families and versions of these families truncated at 0. If \(x_1, \ldots, x_m\) is a random sample from the discrete density (9.13) and \(q(\theta)\) is log-concave, then devise an MM algorithm with updates
\[
\theta_{n+1} = \frac{\bar{x} q(\theta_n)}{q'(\theta_n)},
\]
where \(\bar{x}\) is the sample average of the observations \(x_i\).}

\subsubsection*{Q18 Consider a random sample \( x_1, \ldots, x_m \) from a two-parameter Weibull density
\[
f(x) = \frac{\kappa}{\lambda} \left( \frac{x}{\lambda} \right)^{\kappa - 1} e^{-(x / \lambda)^\kappa}
\]
over the interval \((0, \infty)\). Show that the maximum of the loglikelihood with respect to \(\lambda > 0\) for \(\kappa > 0\) fixed satisfies \(\lambda^\kappa = \frac{1}{m} \sum_{i=1}^m x_i^\kappa\) [6]. Derive the profile loglikelihood
\[
m \ln \kappa - m \ln \left( \sum_{i=1}^m e^{\kappa \ln x_i} \right) + (\kappa - 1) \sum_{i=1}^m \ln x_i + \text{constant}
\]
by substituting this \(\lambda\) into the ordinary loglikelihood. Since the second term of the profile loglikelihood is concave in \(\kappa\), one can apply the majorization (9.3). Prove that this leads to the MM algorithm
\[
\frac{1}{\kappa_{n+1}} = \frac{\sum_{i=1}^m x_i^{\kappa_n} \ln x_i}{\sum_{i=1}^m x_i^{\kappa_n}} - \frac{1}{m} \sum_{i=1}^m \ln x_i.
\]}

\subsubsection*{Q19 The Rasch model of test taking says that person \(i\) gives a correct response to question \(j\) of an exam with probability
\[
\frac{e^{\alpha_i + \beta_j}}{1 + e^{\alpha_i + \beta_j}}.
\]
Justify the loglikelihood
\[
L(\alpha, \beta) = \sum_{i=1}^p x_i \alpha_i + \sum_{j=1}^s y_j \beta_j - \sum_{i=1}^p \sum_{j=1}^s \ln(1 + e^{\alpha_i + \beta_j}),
\]
where \(x_i\) is the number of correct answers given by person \(i\), and \(y_j\) is the number of correct answers of question \(j\) given by all test takers. Based on the previous problem, majorize \(-L(\alpha, \beta)\) by a quadratic. Minimizing the quadratic can be accomplished by one step of Newton’s method. Thus, the Rasch model succumbs to a straightforward MM algorithm.}

\newpage
\section*{Chapter 10: Data Mining}

\subsubsection*{Q3: In the EM clustering model, suppose we relax the assumption that the cluster distributions share a common covariance matrix \(\Omega\). This can cause the likelihood function to be unbounded. Imposing a prior stabilizes estimation \([28]\). It is mathematically convenient to impose independent inverse Wishart priors on the different covariance matrices \(\Omega_j\). This amounts to adding the logprior
\[
- \sum_{j=1}^{k} \left[ \frac{a}{2} \ln \det \Omega_j + \frac{b}{2} \operatorname{tr}(\Omega_j^{-1} S_j) \right]
\]
to the loglikelihood, where the positive constants \(a\) and \(b\) and the positive definite matrices \(S_j\) must be determined. Show that imposition of this prior does not change the MM updates of the fractions \(\pi_j\) or the cluster centers \(\mu_j\). The most natural choice is to take all \(S_j\) equal to the sample variance matrix
\[
S = \frac{1}{m} \sum_{i=1}^{m} (y_i - \bar{y})(y_i - \bar{y})^*.
\]
Show that the MM updates of the \(\Omega_j\) are now
\[
\Omega_{n+1,j} = \frac{a}{a + \sum_{i=1}^{m} w_{nij}} \left( \frac{b}{a} S \right) + \frac{\sum_{i=1}^{m} w_{nij}}{a + \sum_{i=1}^{m} w_{nij}} \tilde{\Omega}_{n+1,j}
\]
\[
\tilde{\Omega}_{n+1,j} = \frac{1}{\sum_{i=1}^{m} w_{nij}} \sum_{i=1}^{m} w_{nij} (y_i - \mu_{n+1,j})(y_i - \mu_{n+1,j})^*.
\]
In other words, the penalized EM update is a convex combination of the standard EM update and the mode \(\frac{b}{a} S\) of the prior. Chen and Tan \([28]\) tentatively recommend the choice \(a = b = \frac{2}{\sqrt{m}}\).}

\subsubsection*{Q4: The mean shift algorithm is a technique for finding the local maxima of an empirically constructed probability density \([55, 59]\). The algorithm has applications in computer vision and image processing. Suppose \( y_1, \ldots, y_m \) is a random sample of points from some unknown density function. One can approximate the density by a weighted sum
\[
f(x) = \sum_i w_i h(\|x - y_i\|^2)
\]
of shifted versions of a density \( h(\|x\|^2) \). Let us assume that \( h(s) \) is nonnegative, decreasing, convex, and differentiable on \([0, \infty)\). The mean shift algorithm looks for a local maximum of \( f(x) \) in a neighborhood of an initial point \( x_0 \) and iterates according to
\[
x_{n+1} = \frac{\sum_i w_i h'(\|x_n - y_i\|^2) y_i}{\sum_i w_i h'(\|x_n - y_i\|^2)}.
\]
Prove that the iterates constitute an MM algorithm for maximizing \( f(x) \). What are the iterates in the Gaussian setting \( h(s) = e^{-cs} \)?}

\subsubsection*{Q6: In MCDA there are an infinity of regular simplexes. One appealing simplex has vertices
\[
\mathbf{v}_j = 
\begin{cases} 
(c - 1)^{-1/2} \mathbf{1} & \text{if } j = 1 \\ 
r \mathbf{1} + s \mathbf{e}_{j-1} & \text{if } 2 \leq j \leq c, 
\end{cases}
\]
where
\[
r = \frac{-1 + \sqrt{c}}{(c - 1)^{3/2}}, \quad s = \sqrt{\frac{c}{c - 1}},
\]
and \(\mathbf{e}_j\) is the \(j\)th standard coordinate vector in \(\mathbb{R}^{c-1}\). Show that these choices place the \(c\) vertices on the surface of the unit sphere in \(\mathbb{R}^{c-1}\) and that all vertex pairs are equidistant. Any rotation, dilation, or translation of these vectors preserves the equidistant property.}

\subsubsection*{Q7: Demonstrate that it is impossible to situate \(c + 1\) points in \(\mathbb{R}^{c-1}\) so that all pairs of points are equidistant under the Euclidean norm [174].}

\subsubsection*{Q8: If the matrix \(A\) has full svd \(U \Sigma V^*\), then the best rank \(r\) approximation of \(A\) in the Frobenius norm is
\[
B = \sum_{j=1}^r \sigma_j u_j v_j^*
\]
based on the columns of \(U\) and \(V\) and diagonal entries of \(\Sigma\). This is the content of the Eckart–Young theorem [53] as proved in Proposition A.6.4. Given this result, use existing utilities in Julia to write a function for matrix completion.}

\subsubsection*{Q9: In the Kullback–Leibler divergence version of NMF, suppose the entries \(y_{ij}\) of \(\mathbf{Y}\) are integers. They can then be viewed as realizations of independent Poisson random variables with means \(\sum_k u_{ik}v_{kj}\). In this setting the loglikelihood is
\[
L(U, V) = \sum_i \sum_j \left[ y_{ij} \ln \left( \sum_k u_{ik}v_{kj} \right) - \sum_k u_{ik}v_{kj} \right].
\]
\noindent Maximization with respect to \(U\) and \(V\) should lead to a good factorization. Verify the minorization
\[
\ln \left( \sum_k u_{ik}v_{kj} \right) \geq \sum_k \frac{a_{nikj}}{s_{nij}} \ln \left( \frac{s_{nij}}{a_{nikj}} u_{ik}v_{kj} \right),
\]
\noindent where
\[
a_{nikj} = u_{nik}v_{nkj}, \quad s_{nij} = \sum_k a_{nikj},
\]
\noindent and \(n\) indicates the current iteration. Given this minorization, derive the alternating multiplicative updates
\[
u_{n+1,ik} = u_{nik} \frac{\sum_j y_{ij} \frac{v_{nkj}}{s_{nij}}}{\sum_j v_{nkj}}
\]
\noindent and
\[
v_{n+1,kj} = v_{nkj} \frac{\sum_i y_{ij} \frac{u_{nik}}{s_{nij}}}{\sum_i u_{nik}}.
\]}

\subsubsection*{Q10: In problem (9) calculate the partial derivative}
\[
\frac{\partial}{\partial u_{il}} L(U, V) = \sum_j v_{lj} \left( \frac{y_{ij}}{\sum_k u_{ik}v_{kj}} - 1 \right).
\]
\noindent Show that the conditions \(\min\{u_{il}, - \frac{\partial}{\partial u_{il}} L(U, V)\} = 0\) for all pairs \((i, l)\) are both necessary and sufficient for \(U\) to maximize \(L(U, V)\) when \(V\) is fixed. The same conditions apply in minimizing the Frobenius criterion \(\|\mathbf{Y} - \mathbf{U} \mathbf{V}\|_F^2\) with different partial derivatives.

\subsubsection*{Q11: In the matrix factorization, it may be worthwhile shrinking the estimates of the entries of \(U\) and \(V\) toward 0 [140]. Let \(\lambda\) and \(\mu\) be positive constants, and consider the penalized objective functions
\[
l(U, V) = L(U, V) - \lambda \sum_i \sum_k u_{ik} - \mu \sum_k \sum_j v_{kj}
\]
\[
r(V, W) = \|\mathbf{Y} - \mathbf{UV}\|_F^2 + \lambda \sum_i \sum_k u_{ik}^2 + \mu \sum_k \sum_j v_{kj}^2
\]
\noindent with lasso and ridge penalties, respectively. Here the Poisson loglikelihood \(L(U, V)\) is studied in problem (9). Derive block optimization updates for these objective functions. Explain why the updates maintain positivity and why shrinkage is obvious, with stronger shrinkage for the lasso penalty with small parameters.}

\subsubsection*{Q12: Find the Euclidean projection operators for the four following closed convex sets in \(\mathbb{R}^p\): (a) closed ball \(\{x : \|x - z\| \leq r\}\), (b) closed rectangle \(\{x : a \leq x \leq b\}\), (c) hyperplane \(\{x : a^* x = b\}\) for \(a \neq 0\), and (d) closed halfspace \(\{x : a^* x \leq b\}\).}

\subsubsection*{Q13: Let \(\mathrm{Sym}_n\) denote the subspace of \(n \times n\) symmetric matrices. Find the Euclidean projection mapping an arbitrary \(n \times n\) matrix \(\mathbf{X}\) onto \(\mathrm{Sym}_n\).}

\subsubsection*{Q14: Let \(\mathrm{Skew}_n\) denote the subspace of \(n \times n\) skew-symmetric matrices. Show that the map \(\mathbf{X} \mapsto \frac{1}{2} (\mathbf{X} - \mathbf{X}^*)\) constitutes projection onto \(\mathrm{Skew}_n\) and that the subspaces \(\mathrm{Sym}_n\) and \(\mathrm{Skew}_n\) are orthogonal complements under the Frobenius inner product.}

\subsubsection*{Q15: Let \(\mathrm{Pos}_n\) denote the set of \(n \times n\) positive semidefinite matrices. Find the Euclidean projection mapping an \(n \times n\) symmetric matrix \(\mathbf{X}\) onto \(\mathrm{Pos}_n\). (Hint: The Frobenius norm is invariant under multiplication of its argument by an orthogonal matrix.)}

\subsubsection*{Q16: Let \(S_r\) be the set whose points have at most \(r\) nonzero coordinates. Show that projecting \(\mathbf{y}\) onto \(S_r\) amounts to sorting the coordinates of \(\mathbf{y}\) by magnitude, saving the \(r\) largest, and sending the remaining \(n - r\) coordinates to zero. For what \(\mathbf{y}\) is the projection multivalued?}

\subsubsection*{Q17: The set \(U_k^n = \{0,1\}^n \cap \{x \in \mathbb{R}^n : \mathbf{x}^* \mathbf{1} = k\}\) is the collection of \(\binom{n}{k}\) vertices of the unit cube whose coordinates sum to \(k\). Show that projection of \(\mathbf{y}\) onto \(U_k^n\) replaces the \(k\) largest entries of \(\mathbf{y}\) by \(1\) and the remaining entries by \(0\).}

\subsubsection*{Q18: According to Geman and McClure [62], robust regression can be achieved by minimizing the following amendment
\[
f(\beta) = \sum_{i=1}^{n} \frac{(y_i - \mathbf{x}_i^* \beta)^2}{c^2 + (y_i - \mathbf{x}_i^* \beta)^2}
\]
of the usual least linear regression criterion. Based on the concavity of the function \(s \mapsto \frac{s}{c^2 + s}\), devise a weighted least squares MM algorithm for minimizing \(f(\beta)\).}

\subsubsection*{Q19: Consider the piecewise linear function \( f(\mu) = c\mu + \sum_{i=1}^{n} w_i |y_i - \mu| \), where the points \( y_i \) satisfy \( y_1 < y_2 < \cdots < y_n \) and the positive weights satisfy \( \sum_{i=1}^{n} w_i = 1 \). Show that \( f(\mu) \) has no minimum when \( |c| > 1 \). What happens when \( c = 1 \) or \( c = -1 \)? This leaves the case \( |c| < 1 \). Show that a minimum occurs when
\[
\sum_{y_i > \mu} w_i - \sum_{y_i \leq \mu} w_i \leq c \quad \text{and} \quad \sum_{y_i \geq \mu} w_i - \sum_{y_i < \mu} w_i \geq c.
\]
The case \( c = 0 \) produces the weighted median. (Hints: A crude plot of \( f(\mu) \) might help. What conditions on the right-hand and left-hand derivatives of \( f(\mu) \) characterize a minimum?)}

\newpage
\section*{Chapter 11: Fast Fourier Transfrom}

\newpage
\section*{Chapter 12: Monte Carlo Methods}

\newpage
\section*{Chapter 13: Neural Networks}



\end{document}
